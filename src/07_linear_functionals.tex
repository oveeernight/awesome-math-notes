% !TeX root = ./document.tex
\documentclass[document]{subfiles}
\begin{document}
\part{Линейные функционалы}
\chapter{Геометрический смысл линейного функционала}
Линейное пространство, без нормы, без топологии, может, уже даже в алгебре доказывали такую теорему.

\begin{theorem}
    $X$ --- линейное пространство над $\bC$ ($\bR$)
    \begin{enumerate}
        \item $f \in \Lin(X, \bC), f \ne \bZero, L = \Ker f \Rightarrow$ 
        \[ \dim (X / L) = 1 \]
        $\codim L \coloneqq \dim (X / L)$ --- коразмерность, не то чтобы мы будем этим пользоваться, просто сообщение по секрету
        \item пусть $L \subset X, L$ --- подпространство, такое что $\dim (X/L) = 1$. $x_0 \in X \setminus L \Rightarrow \: \exists! f \in \Lin(X, \bC), 
        L = \Ker f, f(x_0) = 1$
    \end{enumerate}
\end{theorem}

Поскольку образ одномерен, это и означает, что фактор по ядру имеет такую же размерность, а образ у нас это $\bC$

\begin{proof}[1 утверждение]
    Пусть $x_0 \in X \setminus L \Rightarrow f(x_0) \ne 0, v = \frac{x_0}{f(x_0)} \Rightarrow f(v) = 1$
    \begin{gather*}
        (X/L) = \seq{\overline{x}}_{x \in X}, \overline{x} = \seq{x + y | y \in L} \\
        \text{возьмём какой-то } x \in X, \alpha \coloneqq f(x), f(\alpha v) = \alpha f(v) = \alpha \\
        \Rightarrow f(x-\alpha v) = 0 \Rightarrow x - \alpha v \in L \Rightarrow \overline{x} = \alpha \overline{v} \\
        \Rightarrow \dim(X/L) = 1
    \end{gather*}
\end{proof}

\begin{proof}[2 утверждение]
    \begin{gather*}
        (X/L) = \seq{\overline{x}}_{x \in X} \dim(X / L) = 1 \Rightarrow \: \forall x \in X \: \exists \alpha \in \bC : \overline{x} = \alpha \overline{x_0} \\
        \text{определим } f: X \rightarrow \bC \\
        \text{установили, что } \forall x \: \exists \alpha \in \bC : \overline{x} = \alpha \overline{x_0}, \: f(x) \coloneqq \alpha \Rightarrow f \in \Lin(X, \bC) \\
        f(x_0) = 1 \\
        \text{пусть } f(x) = 0 \Rightarrow \overline{x} = 0 \cdot \overline{x_0} = \overline{0} = L \Rightarrow x \in L \Rightarrow \\
        \Ker f = L 
        \intertext{Проверим единственность:}
        \text{пусть } g \in \Lin(X, \bC), \Ker g = L, g(x_0) = 1 \\
        \forall x \in X \: x = y + \alpha x_0 \text{ где } y \in L, \alpha \in \bC \\
        \Rightarrow f(x) = \alpha, g(x) = \alpha
    \end{gather*}
\end{proof}

докажем теперь что-то с функционалами для нормированного пространства

\begin{theorem}[норма линейного функционала]
    $(X, \norm{\cdot})$ --- нормированное пространство. $f \in X^*, f \ne \bZero, L = \Ker f, f(x_0) = 1 \Rightarrow \norm{f} = \frac{1}{\rho(x_0,L)}$
\end{theorem}

\begin{proof}
    \begin{gather*}
        L = f^{-1}(0) \Rightarrow L \text{ --- замкнутое}\\
        d = \rho(x_0, L) = \inf_{y \in L} \norm{x_0 -y} \\
        1 = f(x_0) = f(x_0-y) \Rightarrow \abs{f(x_0-y)} \leq \norm{f} \cdot \norm{x_0 -y} \: \forall y \in L \\
        \Rightarrow 1 \leq \norm{f} \inf_{y \in L} \norm{x_0 - y} = \norm{f} \cdot d \Rightarrow \frac{1}{d} \leq \norm{f}
    \end{gather*}
    Получили неравенство в одну сторону. Теперь в другую:
    \begin{gather*}
        x \notin L \Rightarrow f(x) \ne 0, \: f\left( \frac{x}{f(x)} \right) = 1, f(x_0) = 1 \Rightarrow \\
        f \left(\frac{x}{f(x)} -x_0 \right) = 0 \Rightarrow \frac{x}{f(x)} - x_0 = y, y \in L \\
        \Rightarrow \frac{x}{f(x)} = x_0 -(-y) \Rightarrow \norm{ \frac{x}{f(x)}} = \norm{x_0 -(-y)} \geq d \\
        \Rightarrow \abs{f(x)} \leq \frac{1}{d} \cdot \norm {x} \Rightarrow \norm{f} \leq \frac{1}{d}
    \end{gather*}
    Вот и получили, что было обещано: $\norm{f} = \frac{1}{d}$
\end{proof}

\begin{remark}
    В условиях теоремы, $M = f^{-1}(1)$, тогда $M = x_0 + L, \rho(x_0, L) = \rho(0, M)$. Вместо того, чтобы рассматривать ядро, можно рассматривать такое <<сдвинутое ядро>>.
    Подпространство $L$ можно сдвинуть на  вектор, это довольно очевидно, не будем это доказывать.
\end{remark}

\section{Продолжение линейного функционала}

Новый раздел, в котором наконец появится существенная теорема, до этого были так...

Будет задан функционал с дополнительным условием, и мы будем продолжать его на всё пространство так, чтобы условие сохранилось. Нам понадобится не только анализ, но и математическая логика, в частности, лемма Цорна. Поскольку 
нам никто её не рассказывал, придётся провести ликбез.
Нам понадобится индукция: но не обычная, ведь у нас какие-то гигантские пространства, переход от $n$ к $n+1$ нам ничем не поможет, нужен более хитрый трюк.

\begin{definition}[частично упорядоченное множество]
    $\Rho$ \textbf{ частично упорядоченное множество}, если $\R \subset \Rho \times \Rho, (a,b) \in \R$, то есть $a \leq b$. $\R$ --- порядок, если выполнены аксиомы 
    \begin{enumerate}
        \item $\forall a \in \Rho, (a,a) \in \R$, то есть $a \leq a$ (рефлексивность)
        \item если $(a \leq b \land b \leq c) \Rightarrow a \leq c$ (транзитивность)
        \item если $(a \leq b \land b \leq a)$, то $a = b$ (антисимметричность)
    \end{enumerate}
    важно, что не для всех элементов определён порядок, а для каких-то
\end{definition}

\begin{definition}[линейно упорядоченное множество]
    $\Rho$ --- частично упорядоченное, $A \subset \Rho$, $A$ --- линейно упорядочено, если $\forall a,b \in A, a \leq b$ или $b \leq a$
\end{definition}

\begin{definition}[верхняя грань множества]
    $A \subset \Rho$, $x$ --- верхняя грань для $A$, если $a \leq x \: \forall a \in A$
\end{definition}

\begin{definition}[максимальный элемент множества]
    $y$ --- максимальный элемент в $\Rho$, если $y \leq a \Rightarrow y = a$.
    Максимальный в том смысле, что больше него не существует, но таких максимумов может быть хоть миллион, и они между собой не сравнимы.
\end{definition}


\begin{lemma*}[Цорн]
    Если в $\Rho$ любое линейно упорядоченное множество имеет верхнюю грань, то в $\Rho$ есть максимальный элемент
\end{lemma*}

\begin{axiom*}[Выбора]
    $\seq{B_\alpha}_{\alpha \in A}$, $B_\alpha \ne \bZero \Rightarrow \: \exists C = \{ b_\alpha: b_\alpha \in B_\alpha \}_{\alpha \in A}$
\end{axiom*}
Если есть алгоритм выбора элементов из множества, то пользуемся им, без этой аксиомы. \\

Для общего развития: Аксиома Выбора $\Leftrightarrow$ Лемма Цорна. \\ 
Закончили с ликбезом по теории множеств.

\begin{definition}[выпуклый функционал]
    $X$ --- линейное пространство над $\bC$ ($\bR$). $p: x \rightarrow \bR, p$ --- выпуклый функционал, если 
    \begin{enumerate}
        \item $p(x+y) \leq p(x) + p(y) \: \forall x,y \in X$ 
        \item $p(tx) = tp(x) \: \forall t \geq 0$
    \end{enumerate} 
\end{definition}

\begin{remark}
    $p$ --- полунома, тогда $p(\lambda x) = \abs{\lambda} p(x) \: \forall \lambda \in \bC (\bR) \Rightarrow p$ --- выпуклый функционал
\end{remark}

Считается, что весь линейный функциональный анализ стоит на трёх китах, и мы дошли до Кита №1.
\begin{theorem}[Хан-Банах, о продолжении линейного функционала в вещественном пространстве]
    $X$ --- линейное пространство над $\bR$, $p: X \rightarrow \bR$, $p$ --- выпуклый функционал.
    $L \subset X, L$ --- подпространство, $f \in \Lin(L, \bR), f(x) \leq p(x) \: \forall x \in L$ (говорят $f$ подчинён $p$)
    \[ \exists g \in \Lin(X, \bR), g(x) = f(x), x \in L \quad g(x) \leq p(x) \: \forall x \in X \]
\end{theorem}
Тут очень важно, что пространство вещественное, у нас будет другая теорема для комплексного. Эта теорема всё время возникает, мы ей либо по умолчанию пользуемся, либо следствиями из неё.

Доказательство будет состоять из 2 частей. Первая --- естественная часть МА: покажем, что существует функционал, продлённый на одну размерность больше и который совпадает с $f$ на подпространстве. Во второй части продлим на всё $X$, там нам и понадобится это логическое жульничество.
\begin{proof}
   \begin{gather*}
        f \in \Lin(L, \bR), z \in X \setminus L \\
        L_1 = \calL(L, z) = \seq{x + tz : t \in \bR, x \in L} \\
        \intertext{докажем, что $\exists f_1 \in \Lin(L_1, \bR): f_1 |_L = f, f_1(y) \leq p(y) \: \forall y \in L_1$; мы можем распоряжаться только значением $f_1$}
        f_1(z) = c \quad c \in \bR, \text{ выберем <<с>> так, как надо } \\
        y = x + tz \in L_1 \Rightarrow f_1(y) = f(x) + tc \\
        \intertext{хотим доказать $f(x) + tc \leq p(x+tz) \forall t \in \bR$, напишем 2 неравенства для положительных и отрицательных $c$ соответственно, потому что из функционала выносить можно только положительные числа} 
        \begin{cases}
            f(x) + tc \leq p(x + tz) & t > 0 \\
            f(x) - tc \leq p(x - tz) & t > 0
        \end{cases} \Leftrightarrow \\
        \begin{cases}
            f \left( \frac{x}{t} \right) + c \leq p \left( \frac{x}{t} + z \right) & \forall t > 0 \\
            f\left(\frac{x}{t}\right)-c \leq p \left( \frac{x}{t} -z \right)  & \forall t > 0
        \end{cases}, \frac{x}{t} \in L \Leftrightarrow x \in L \\
        u = \frac{x}{t}, u \in L, v = \frac{x}{t} \Leftrightarrow \left. \begin{matrix}
            f(u) + c \leq p(u+z) \\
            f(v) - c \leq p(v-z)
        \end{matrix}  \right\} \Leftrightarrow \\
        f(v) - p(v-z) \leq c \leq p(u+z) - f(u), \: u,v \in L 
        \intertext{если такое $c$ есть, все хорошо, а если нет --- ужасно}
        \text{обозначим } A = \seq{f(v) - p(v-z) : v \in L} \subset \bR, B = \seq{p(u+z) - f(u): u \in L } \subset \bR \\
        \intertext{проверим, что $\forall a \in A, \forall b \in B \: a \leq b$. Это и будет означать, что между этими множествами и есть какой-то элемент (из-за полноты вещественной прямой)}
        f(v) - p(v-z) \leq p(u+z) - f(u) \Leftrightarrow f(v) + f(u) \leq p(u+z) + p(v-z) \\
        f(v) + f(u) = f(u+v) \leq [[u+v \in L]] p(u+v) \: [[\text{выпуклость } p ]]\: \leq p(u+z) + p(v-z) \\
        \Rightarrow \: \exists c \in \bR: f_1(z) = c \Rightarrow f_1(y) \leq p(y) \: \forall y \in L_1, f_1|_L = f
   \end{gather*}
   итак, мы продолжили функционал на размерность+1, и
   если бы было сепарабельное или банахово пространство, мы бы ограничились обычной индукцией, увеличивая размерность на 1, и по непрерывности пришли бы к пределу, и замыкание было бы всем $X$.
   Но раз у нас всего этого нет, мы будем пользоваться леммой Цорна, которая по всем кардиналам эквивалентна трансфинитной индукции.
   Что же у нас тут будет частично упорядоченным множеством? Рассмотрим все возможные продолжения линейного фунционала, удовлетворяющие условиям
   \[ \Rho = \seq{(M,h)} \] 
   где $L \subset M$ --- подпространство $X$, $h \in \Lin(M, \bR), h|_L = f, h(x) \leq p(x) \: \forall x \in M$. Докажем, что $\exists M = X$, то есть 
   $(X, h) \in \Rho$. Раз в множестве $\Rho$ есть максимальный элемент, то он равен $X$, вот такой краткий план.

   Как определяется частичный порядок в $\Rho$? 
   $(M_1,h_1) \leq (M_2,h_2)$, если $M_1 \subset M_2, {h_2}|_{M_1} = h_1$ 

   $\seq{(M_\alpha, h_\alpha)}_{\alpha \in A}$ --- линейно упорядоченное множество. 
   Построим верхнюю грань: 
   \[ M_0 = \bigcup_{\alpha \in A} M_\alpha, h_0: M_0 \rightarrow \bR \]
   пусть $x \in M_0 \Rightarrow \: \exists \alpha \in A : x \in M_\alpha, h_0(x) \coloneqq h_\alpha(x)$
   и то, и другое определение требует обоснования корректности, ведь объединение подпространств не обязано быть подпространством (на вещественной плоскости: объединение 2 прямых, проходящих через 0 --- непонятно, что вообще такое).
   Проверим, что $M_0$ --- подпространство
   \begin{gather*}
        \text{пусть } x,y \in M_0 \Rightarrow \: \exists \alpha, \beta \in A : x \in M_\alpha, y \in M_\beta \\
        \intertext{вспоминаем про линейный порядок}
        (M_\alpha, h_\alpha) \leq (M_\beta, h_\beta) \text{ или }  (M_\beta, h_\beta) \leq (M_\alpha, h_\alpha) \\
        \text{ пусть } (M_\alpha, h_\alpha) \leq (M_\beta, h_\beta) \Rightarrow M_\alpha \subset M_\beta \Rightarrow x \in M_\beta \Rightarrow \lambda x + \mu y \in M_\beta \\
        \Rightarrow \lambda x + \mu y \in M_0 \Rightarrow M_0 \text{ подпространство }
        \intertext{проверим корректность определения $h_0$, то есть что оно не должно зависеть от того, возьмём мы $\alpha$ или $\beta$}
        \text{пусть } x \in M_0, \text{ пусть } x \in M_\alpha, x \in M_\beta, \text{ пусть } (M_\alpha, h_\alpha) \leq (M_\beta, h_\beta) \text{  или наоборот } \\
        \Rightarrow h_\alpha(x) = h_\beta(x) \Rightarrow  \left.\begin{matrix}
            h_0(x) = h_\alpha(x) \\
            h_0(x) = h_\beta(x)
        \end{matrix}  \right\}
        \intertext{$h_\alpha(x) = h_\beta(x)$, потому что если выберем для $h_0$ $h_\beta$, то по определению $h_\beta|_{M_\alpha} = h_\alpha$. В итоге $h_0$ определено корректно, одно другому не противоречит}
        h_0(x) \leq p(x) \: \forall x \in M_0 \text{ (очевидно) } \Rightarrow (M_0, h_0) \in \Rho \\
        \alpha \in A \quad (M_\alpha, h_\alpha) \leq (M_0,h_0) \text{ --- верхняя грань} 
        \intertext{теперь, когда мы рассмотрели произвольное линейное упорядоченное множество и доказали, что у него есть верхняя грань, мы можем применить лемму Цорна}
        \Rightarrow \text{ в } \Rho \: \exists \text{ максимальный элемент } (M,h) \in \Rho \\
        \text{пусть } M \subsetneq X \: \exists z \in X \setminus M, M_1 = \Lin(M,z) \\
        \intertext{построим как в первой части продолжение $(M_1,f_1) \in \Rho$}
        (M,h) \leq (M_1,f_1), M \subsetneq M_1 \text{ противоречит максимальности } (M,h) \\
        \Rightarrow M = X, (M,h) \text{ --- искомое продолжение}
   \end{gather*}
\end{proof}

Прежде, чем рассказать комплексный аналог, сначала применение вещественного случая.

\begin{theorem}[обобщённый предел ограниченной последовательности]
    \begin{gather*}
        l^\infty_{\bR} = \seq{x = \seq{x_n}^\infty_{n=1}, x_n \in \bR, \norm{x} = \sup_{n \in \bN} \abs{x_n} < + \infty} \\
        \Rightarrow \: \exists F \in \B(l^\infty, \bR) = (l^\infty)^* \\
        \forall x \in l^\infty \: \underline{\liml} x_n \leq F(x) \leq \overline{\liml} x_n
    \end{gather*}
    в частности, если $\exists \liml_{n \to \infty} x_n = x_0$, то $F(x) = x_0$
\end{theorem}
То есть каждой ограниченности сопоставляется число, причём это отображение линейное.

\begin{proof}
    \begin{gather*}
        x \in l^\infty, p(x) \coloneqq \overline{\liml} x_n, x = \seq{x_n}^\infty_{n=1} \in l^\infty 
        \intertext{откуда же берётся неравенство треугольника, которое фигурирует в выпуклости? Когда-то в детстве мы доказывали такое неравенство, оно даже в Демидовиче есть}
        \overline{\liml} (x_n + y_n) \leq \overline{\liml} x_n + \overline{\liml} y_n
    \end{gather*}
    \textit{напоминание, как это доказывается через альтернативное определение верхнего предела}
    \begin{gather*}
        a_n = \sup \{x_n, x_{n+1}, \ldots \}, a_n \text{ убывают к } a, \liml_{n \to \infty} a_n = a, a = \overline{\liml} x_n. \\
        b_n = \sup_{k \geq 0} \{ y_{n+k} \}, b_n \text{ убывают к } b, b = \overline{\liml} y_n. \\
        c_n = \sup_{k \geq 0} \{x_{n+k} + y_{n+k} \}, c_n \text{ убывают к } c = \overline{\liml} (x_n + y_n) \\
        \text{пусть} k \geq 0 \quad x_{n+k} + y_{n+k} \leq a_n + b_n \: \forall k \Rightarrow c_n \leq a_n + b_n \Rightarrow c \leq a + b
    \end{gather*}
    \textit{напоминание закончилось}

    Вот мы доказали, что это функционал 
    \begin{gather*}
        c = \seq{ x = \seq{x_n}^\infty_{n=1}, \: \exists \liml_{n \to \infty} x_n = x_0} \\
        g: c \rightarrow \bR \quad x = \seq{x_n}^\infty_{n=1} \in c \Rightarrow g(x) = x_0 \\
        g(x) = \liml_{n \to \infty} x_n \leq p(x)  \coloneqq \overline{\liml} x_n \\
        \text{по теореме Хана-Банаха } \exists F: l^\infty \rightarrow \bR, F(x) \leq p(x) \\
        F(x) = g(x) = x_0, \text{ если } x \in c \\
        x \in l^\infty, p(-x) = \overline{\liml} (-x_n) = - \underline{\liml} x_n 
        \intertext{\textit{почему это так? представьте последовательность, у которой два предела: нижний --- 1, верхний --- 2, проотрицаем последовательность, 
        получим пределы в -1 и -2, её верхний предел -1 это как раз нижний предел исходной последовательности}}
        -F(x) = F(-x) \leq p(-x) =- \underline{\liml} x_n \Rightarrow F(x) \geq \underline{\liml} x_n
    \end{gather*}
    В формулировке обещалось $\norm{F}=1$. мы можем взять $x = (1, 1, 1, \ldots)$
        \begin{gather*}
        F(x) = 1, \norm{x}_\infty = 1 \Rightarrow \norm{F} \geq 1 \\
        \forall x \: \abs{F(x)} \leq \overline{\liml} x_n \leq \sup x_n = \norm{x}_\infty \Rightarrow \norm{F} \leq 1
        \end{gather*}
\end{proof}

Хочется последнее неравенство записать в более общем случае.
\begin{statement}
    \begin{enumerate}       
        \item $X$ --- линейное, $p(x)$ --- выпуклый функционал, $f \in \Lin(X, \bR)$
        \[ f(x) \leq p(x) \Rightarrow f(x) \geq -p(-x) \] 
        \item если $p(x)$ полунорма, $f(x) \leq p(x) \: \forall x \in X \Rightarrow \abs{f(x)} \leq p(x)$
    \end{enumerate}
\end{statement}

\begin{proof}
    \begin{enumerate}
        \item $f(x) \leq p(x) \Rightarrow f(-x) \leq p(-x) \Rightarrow -f(x) \leq p(-x) \Rightarrow f(x) \geq -p(x) $
        \item $p$ --- полунорма $\Rightarrow p(-x) = p(x) \Rightarrow -p(x) \leq f(x) \leq p(x) \Rightarrow \abs{f(x)} \leq p(x)$
    \end{enumerate}
\end{proof}

Теперь, как было обещано, вариант теоремы продолжения линейного функционала для комплексного случая.

\begin{theorem}[Боненблюст-Собчик, продолжение линейного функционала в комплексном линейном пространстве]
    $X$ над $\bC$. В вещественном случае предполагали что $p$ --- выпуклый функционал, теперь предполагаем чуть большее: $p: X \rightarrow \bR, \: p$ ~--- полунорма,
    $L \subset X$ --- подпространство, $f \in \Lin(L, \bC)$. Второе отличие состоит в том, что мы говорим $\abs{f(x)} \leq p(x) \: \forall x \in L \Rightarrow$ 
    \[ \exists g \in \Lin(X, \bC), g |_L = f, \abs{g(x)} \leq p(x) \: \forall x \in X \]
\end{theorem}

\begin{proof}
    Мы будем использовать доказательство для вещественного случая изо всех сил. Проведём овеществление $X$, то есть $X$ над $\bR$, $x,y \in X, \: a,b \in \bR \Rightarrow ax + by \in X$, то есть забудем на какое-то время,
    что $X$ над $\bC$.
    \begin{gather*}
        f(x) = u(x) + iv(x), \: u,v : L \rightarrow \bR 
        \intertext{Проверим, что $u, v \in \Lin(L, \bR)$, а также покажем что между ними существует связь. Потом примением к $u$ теорему Хана-Банаха, а там, глядишь, и получится то, что требовалось}
        y \in X \Rightarrow f(y) = u(y) + iv(y) \\
        \left. \begin{matrix}
            \Rightarrow f(x) + f(y) = u(x) + u(y) + i(v(x) + v(y)) \\
            f(x+y) = u(x+y) + iv(x+y)
        \end{matrix} \right\} \Rightarrow \\
        u(x+y) = u(x) + u(y) \quad v(x+y) = v(x) + v(y) \\
        \left. \begin{matrix}
            \text{пусть } a \in \bR \Rightarrow f(ax) = u(ax) + iv(ax) \\
            f(ax) = af(x) = a(u(x) + iv(x))
        \end{matrix} \right\} \Rightarrow u(ax) = a(u(x)), v(ax) = av(x)
        \intertext{проверили, что они линейные функционалы в вещественном случае. оказывается, они еще и связаны между собой особым образом}
        f(ix) = if(x) \\
        u(ix) + iv(ix) = i(u(x) + iv(x)) \Rightarrow v(x) = -u(ix) \tag{*}
        \intertext{перед тем, как применять теорему Хана-Банаха проверим, чего меньше этот функционал}
        u(x) \leq \abs{u(x)} \leq \abs{f(x)} \leq p(x) \text{ при } x \in L \\
        \intertext{применяем теорему Хана-Банаха к $u$}
        \exists \underline{\varphi \in \Lin(X, \bR)}, \varphi |_L = u, \varphi(x) \leq p(x) \: \forall x \in X
        \intertext{на всякий случай отметим, что $\abs{\varphi(x)} \leq p(x)$ так как $p$ --- полунорма, вдруг пригодится. По аналогии с (*) определим $\psi$}
        \psi(x) \coloneqq - \varphi(ix) \Rightarrow \underline{\psi \in \Lin(X, \bR)} \quad x \in X \\
        g(x) \coloneqq \varphi(x) + i \psi(x), g |_L = f \Rightarrow g \in \Lin(X, \bR) 
    \end{gather*}
    $g$ линейный в вещественном смысле. Остаётся проверить  что он линейный в комплексном случае (можно вынести $i$) и что он подчинён $p$. Проверяем, что
    $g(ix) = ig(x)$
    \begin{multline*}
        g(ix) = \varphi(ix) + i(-\varphi(-x)) = \varphi(ix) + i\varphi(x) =  i(\varphi(x) - i\varphi(ix)) = \\ =
         i(\varphi(x) + i \psi(x)) = ig(x)
    \end{multline*}
    $\Rightarrow g \in \Lin(X, \bC)$. Теперь проверяем подчинённость
    \begin{gather*}
        \text{пусть } x \in X \: g(x) \in \bC \Rightarrow g(x) = re^{i\theta}, r \geq 0 \Rightarrow
        \intertext{такой трюк: воспользуемся линейностью $g$}
        g(xe^{-i\theta}) = r \\
        r = g(xe^{-i\theta}) = \varphi(xe^{-i\theta}) + i \psi(xe^{-i\theta})
        \intertext{слева у нас вещественное число, а справа комплексное, значит, комплексная часть справа равна нулю. Ещё вспоминаем, что $p$ --- полунорма, и можно вынести модуль любого числа}
        \Rightarrow r = \varphi(xe^{-i\theta}) \leq p(xe^{-i\theta}) = \abs{e^{-i\theta}} \cdot p(x) = p(x) \\
        \abs{g(x)} = r \leq p(x) \quad \forall x \in X
    \end{gather*}
\end{proof}

\section{Продолжение линейных функционалов в нормированном пространстве}

В этой части абсолютно все равно, пространство над $\bR$ или же $ \bC$

\begin{theorem}[Хан-Банах]
    $(X, \norm{\cdot})$ над $\bR \: ( \bC)$. $L \subset X, L$ --- подпространство в алгебраическом смысле,
    $f \in L^* (L^* = \B(L, \bC)) \Rightarrow$ 
    \[ \exists g \in X^*, g |_L = f, \norm{g}_{X^*} = \norm{f}_{L^*} \]
\end{theorem}
Мы уже отмечали, что при продолжении норма может только увеличиться, но в условиях этой теоремы норму же удаётся сохранить.

\begin{proof}
    Если $f = \bZero$, то $g = \bZero$ и так далее
    \begin{gather*}
        \text{пусть } f \ne \bZero, M \coloneqq \norm{f}_{L^*}, p(x) \coloneqq M \cdot \norm{x}, x \in X
        \intertext{$\Rightarrow p$ --- норма ($\Rightarrow$ полунорма $\Rightarrow$ выпуклый функционал)} 
        \text{пусть } x \in L \Rightarrow \abs{f(x)} \leq \norm{f}_{L^*} \cdot \norm{x} = p(x) \text{ (условие подчинения)}
        \intertext{Теперь применяем теорему Хана-Банаха, если $X$ над $\bR$, или Боненблюста-Собчика, если $X$ над $\bC$}
        \exists g \in \Lin(X, \bC) \: (\Lin(X, \bR)) \\
        g |_L = f, \quad \abs{g(x)} \leq p(x) \: \forall x \in X \\
        \Rightarrow \abs{g(x)} \leq M \cdot \norm{x} \: \forall x \in X \Rightarrow \norm{g}_{X^*} \leq M \\
        \Rightarrow \norm{g}_{X^*} \leq \norm{f}_{L^*} \\
        \left(\norm{g}_{X^*} = \sup_{\seq{x \in X: \norm{x} \leq 1}} \abs{g(x)} \geq \sup_{\seq{x \in L: \norm{x} \leq 1}} \abs{f(x)} = \norm{f}_{L^*}\right) \\
        \Rightarrow \norm{g}_{X^*} = \norm{f}_{L^*}
    \end{gather*}
\end{proof}

\begin{corollary}[о достаточном числе линейных функционалов]
    $(X, \norm{\cdot}), x_0 \in X \Rightarrow \: \exists g \in X^*, \norm{g} = 1$, $g(x_0) = \norm{x_0}$, при этом 
    \[ \norm{x_0} = \max \seq{ h(x_0) : h \in X^*, \norm{h} \leq 1} \]
\end{corollary}

\begin{proof}
    Если $x_0 = 0$, то $\forall g \in X^*, \norm{g} = 1 \Rightarrow g(0) = 0$ (при линейном отображении 0 переходит в 0 всегда).

    Пусть $x_0 \ne 0, L = \seq{\alpha x_0 : \alpha \in \bC}$
    \begin{gather*}
        f: L \rightarrow \bC \: f(\alpha x_0) \coloneqq \alpha \norm{x_0} \Rightarrow f \in \Lin(L, \bC) \\
        \abs{f(\alpha x_0)} = \abs{\alpha} \norm{x_0}\Rightarrow \norm{f} = \sup_{\alpha \ne 0} \frac{\abs{f(\alpha x_0)}}{\norm{\alpha x_0}} = 1 \Rightarrow \norm{f}_{L^*} = 1
        \intertext{по теореме Хана-Банаха для нормированного пространства}
        \exists g \in X^*, \norm{g} = 1, g |_L = f \Rightarrow g(x_0) = f(x_0) = \norm{x_0} \\
        \text{пусть } h \in X^*, \norm{h} \leq 1 \Rightarrow \abs{h(x_0)} \leq \norm{h} \cdot \norm{x_0} \leq \norm{x_0} \\
        \Rightarrow \norm{x_0} \geq \sup_{\seq{h \in X^* : \norm{h} \leq 1}} \abs{h(x_0)}, \text{ но } \exists g, \norm{g} = 1, g(x_0) = \abs{x_0} \\
        \Rightarrow \abs{x_0} = \max_{\seq{h \in X^* : \norm{h} \leq 1}} \{ h(x_0) \}
    \end{gather*} 
    в этом смысле и много, то есть есть такой, на котором максимум достигается
\end{proof}

\begin{remark}
    $f \in X^* \Rightarrow \norm{f} = \sup_{\seq{x \in X : \norm{x} \leq 1}} \abs{f(x)}$, то есть максимум может не достигаться
\end{remark}

\begin{example}
    \begin{gather*}
        C[-1,1] = X, \varphi(x) = \begin{cases}
            -1 & -1 \leq x < 0 \\
            1 & 0 \leq x \leq 1
        \end{cases} \\
        G_\varphi(f) = \int^1_{-1} f(x) \varphi(x) dx \quad G_\varphi \in (C[-1,1])^* \\
        \norm{G_\varphi} = \int^1_{-1} \abs{\varphi(x)} dx = 2
    \end{gather*}
    Мы показывали, что норма такого функционала всегда будет больше $2 - \varepsilon  \: \forall \varepsilon$
\end{example}

В качестве упражнения доказать, что $\nexists f \in C[-1,1], \norm{f} \leq 1, \abs{G(f)} = 2$.

\begin{corollary}[расстояние от элемента до подпространства]
    $ (X, \norm{\cdot}), L \subset X, L = \overline{L}$ --- подпространство
    \begin{gather*}
        x_0 \in X, d = \rho(x_0, L) = \inf_{y \in L} \norm{x_0 - y} \Rightarrow \\
        \exists g \in X^*, \norm{g} = 1, g |_L = 0, g(x_0) = d, \text{ при этом } \\
        d = \max \seq{ \abs{h(x_0)}, h \in X^*, \norm{h} \leq 1, h |_L = 0}
    \end{gather*}
\end{corollary}

Это следствие полезно для решения экстремальных задач: от инфимума можно перейти к максимуму и решать другую задачу.

\begin{proof}
    Если $x_0 \in L$, то $d = 0, \: \exists g |_L = 0, \norm{g} = 1 (\text{если } L \ne X)$
    \begin{gather*}
        \text{пусть } x_0 \in X \setminus L, M = \calL(L, x_0) = \seq{\alpha x_0 + y : \alpha \in \bC, y \in L} \\
        f : M \rightarrow \bC, f(\alpha x_0 + y) \coloneqq \alpha \Rightarrow \: \forall y \in L \: f(y) = 0 \\
        f^{-1}(0) = L, f \in \Lin(M, \bC), f(x_0) = 1, \norm{f} = \frac{1}{d}
        \intertext{это уже вычислили в геометрическом смысле линейного функционала}
        f_1 = df \Rightarrow \norm{f_1}_{M^*} = 1, f_1(x_0) = d
        \intertext{по теореме Хана-Банаха для нормированного пространства}
        \exists g \in X^*, \norm{g}_{X^*} = 1, g |_M = f_1 \Rightarrow g(x_0) = d, g |_L = f |_L = 0
        \intertext{это была первая часть утверждения следствия}
        \text{пусть } h \in X^*, \norm{h} \leq 1, h(y) = 0 \: \forall y \in L \Rightarrow \\
        \abs{h(x_0)} = \abs{h(x_0 - y)} \leq \norm{h} \cdot \norm{x_0-y} \leq \norm{x_0-y} \quad \forall y \in L \Rightarrow \\
        \abs{h(x_0)} \leq d \Rightarrow \\
        \sup \{ \abs{h(x_0)} : \norm{h} \leq 1, h|_L = 0 \} \leq d, \text{ но } \exists g  \Rightarrow \\
        d = \max \{ \abs{h(x_0)} : \norm{h} \leq 1, h |_L = 0 \}
    \end{gather*}
\end{proof}

\begin{remark}
    Следствие 1 --- частный случай следствия 2 при $L = \{ 0 \}$. На экзамене можно рассказать только второе следствие, отметив, что первое является его частным случаем
\end{remark}

\begin{corollary}[критерий полноты системы элементом в нормированном пространстве]
    $(X, \norm{\cdot})$ --- нормированное пространство, $x_\alpha \in X, A$ --- множество индексов, $\seq{x_\alpha}_{\alpha \in A}$ --- полное семейство в $X \Leftrightarrow$
    если $f \in X^*, f(x_\alpha) = 0, \alpha \in A \Rightarrow f = \bZero$
\end{corollary}
Критерий проверять гораздо проще, чем определение.
\begin{proof}
    $\Rightarrow$ \\
    \begin{gather*}
        f(x_\alpha) = 0, L = \calL \seq{x_\alpha}_{\alpha \in A}, x \in L \Rightarrow \\
        x = \sum^n_{k=1} c_k x_{\alpha_k} \Rightarrow f(x) = 0 \\
        \text{пусть } z \in X, y_n \in L \: \exists \{y_n\}^\infty_{n=1}, \liml_{n \to \infty} y_n = z \text{ (полнота)}, f \text{ --- непрерывная} \Rightarrow \\
        \liml_{n \to \infty} f(y_n) = f(z) \Rightarrow f(z) = 0 \\
        \Rightarrow f = \bZero
    \end{gather*}
    $\Leftarrow$ \\
    \begin{gather*}
        \seq{x_\alpha}_{\alpha \in A} \text{ --- полная} \Leftrightarrow \overline{L} = X \\
        \text{ пусть } \overline{L} \subsetneq X \Rightarrow \: \exists x_0 \in X \setminus \overline{L} \stackrel{\text{Сл.2}}{\Rightarrow} \: \exists g \in X^* \\
        \underline{g|_{\overline{L}} = 0} (\Rightarrow g = \bZero), g(x_0) = \rho(x_0, \overline{L}) \ne 0 \quad d = \rho(x_0, \overline{L}) \\
        \text{но } \underline{g(x_\alpha) = 0 \: \forall \alpha}, g \ne \bZero
    \end{gather*}
\end{proof}

Наконец, с помощью последнего следствия докажем такую теорему

\begin{theorem}
    $(X, \norm{\cdot})$. Если $X^*$ сепарабельно, то $X$ --- сепарабельно 
\end{theorem}

\begin{proof}
    \begin{gather*}
        \exists \seq{f_n}, f_i \in X^* \text{ --- плотная система в } X^*
        \intertext{вспомним, что $\norm{f_n} = \sup_{\seq{x \in X: \norm{x} = 1}} \abs{f_n(x)}$}
        \Rightarrow \: \exists x_n, \norm{x_n} = 1 \quad \norm{f_n} \geq \abs{f_n(x_n)} \geq \frac{1}{2} \norm{f_n} \\
        \text{проверим, что } \{x_n\}^\infty_{n=1} \text{ --- полная в } X
        \intertext{возьмём  произвольный линейный функционал $f$ и предположим, что он обращается в 0 на всех $x_n$}
        \text{пусть } f \in X^*, f(x_n) = 0  \\
        \text{плотность } \seq{f_n} \Rightarrow \: \exists \seq{f_{n_k}}^\infty_{k=1}\\
        \liml_{k \to \infty} \norm{f-f_{n_k}} = 0 \\
        \underbrace{\abs{(f-f_{n_k})(x_{n_k})}}_{=\abs{f_{n_k}(x_{n_k})}\geq \frac{1}{2}\norm{f_{n_k}}} \leq \norm{f-f_{n_k}} \cdot \underbrace{\norm{x_{n_k}}}_{=1} \underset{k \to \infty}{\longrightarrow} 0 \\
        \Rightarrow \liml_{k \to \infty} \norm{f_{n_k}} = 0 \Rightarrow f = \bZero \\
        \stackrel{\text{Сл.3}}{\Rightarrow} \seq{x_n}^\infty_{n=1} \text{ --- полная} \\
        (E = \seq{\sum^n_{k=1} c_k x_k, c_k \in \bQ, n \in \bN} \text{ --- счётное всюду плотное в } X)
    \end{gather*}
\end{proof}

\end{document}
