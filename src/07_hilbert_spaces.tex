% !TeX root = ./document.tex
\documentclass[document]{subfiles}
\begin{document}
\chapter{Гильбертовы пространства}
Кто-то говорил, что матобесам в курсе ФА надо читать только гильбертовы пространства. Но неизвестно, как жить без трех китов функционального анализа, которые нас ждут дальше :(. 
А вы бы хотели 32 лекции про гильбертовы пространства?

\begin{definition}
    $H$ -- линейное пространство над $\bC$. Скалярное произведение $H \times H \rightarrow \bC$, $x,y \in X, (x,y)$ -- скалярное проихведение удовлетворяет следующим аксиомам
    \begin{enumerate}
        \item $(\lambda x, y) = \lambda(x,y), \lambda \in \bC, \: x,y \in H$ 
        \item $(x, y+z) = (x,y) + (x,z)$
        \item $(y,x) = \overline{(x,y)}$ (комплексное сопряжение) 
        \item $(x,x) \geq 0, (x,x) = 0 \Leftrightarrow x = 0$
    \end{enumerate}
    если $H$ над $\bR$, то 3 выглядит как $(y,x) = (x,y)$
\end{definition}

Снабдим $H$ нормой: $\norm{x} := \sqrt{(x,x)}$ -- норма, порожденная скалярным произведением. $(H, \norm{x})$ называется предгильбертовым пространством.

Если $(H, \norm{\cdot})$ полное, то $H$ -- гильбертово.

\begin{property}[скалярное произведение]
    \begin{enumerate}
        \item $x,y \in H \Rightarrow |(x,y)| \leq \norm{x} \cdot \norm{y}$ (неравенство К-Б)
        \item $\norm{x} = \sqrt{(x,x)}$ удовлетворяет аксиомам нормы
        \item $\norm{x+y}^2 + \norm{x-y}^2 = 2(\norm{x}^2 + \norm{y}^2)$ (тождество параллеллограмма)
        \item непрерывность $(x,y)$, то есть $\liml_{n \to \infty} x_n = x, \liml_{n \to \infty} y_n = y \Rightarrow \liml_{n \to \infty} (x_n, y_n) = (x,y)$
    \end{enumerate}
\end{property}

\begin{proof}[2]
    \begin{gather*}
        \norm{x} = 0 \Leftrightarrow (x,x) = 0 \Leftrightarrow x = 0 \\
        \norm{\lambda x}^2 = (\lambda x, \lambda x) = \lambda \cdot \overline{\lambda} (x,x) = |\lambda|^2 \norm{x}^2
    \end{gather*}
    \begin{multline*}
        \norm{x+y}^2 = (x+y, x+y) = \norm{x}^2 + (x,y) + (y,x) + \norm{y}^2 = \\
        = \norm{x}^2 + 2 \Real (x,y) + \norm{y}^2 \leq \norm{x}^2 + 2 \norm{x} \cdot \norm{y} + \norm{y}^2 = \\
        = (\norm{x} + \norm{y})^2
    \end{multline*}
\end{proof}
Кто не верит в тождество параллелограмма, может проверить сам
\begin{proof}[4]
    \begin{multline*}
        |(x,y) - (x_n, y_n)| \leq |(x,y) - (x, y_n)| + |(x,y_n) - (x_n, y_n)| = \\
        = |(x,y - y_n)| + |(x-x_n, y_n)| \stackrel{\text{К-Б}}{\leq} \\
        \leq \norm{x} \cdot \underbrace{\norm{y-y_n}}_{\to 0} + \underbrace{\norm{x - x_n}}_{\to 0} \underbrace{\norm{y_n}}_{\leq M} \underset{n \to \infty}{\longrightarrow} 0
    \end{multline*}
    $\liml_{n \to \infty} \norm{y_n} = \norm{y} \Rightarrow \: \exists M : \norm{y_n} \leq M $
\end{proof}

\begin{example}
    \begin{gather*}
        l^2_n = \seq{x: x = \seq{x_1, \ldots, x_n}, x_j \in \bC}, \norm{x}_2 = \sqrt{\sum^n_{k=1} \abs{x_j}^2} \\
        (x,y) = \sum^n_{j=1} x_j \overline{x_j}, l^2_n \text{ -- гильбертово} \\
        y = (y_1, \ldots, y_n), y_j \in \bC, \overline{y_j} \text{ -- комплексное сопряжение}
    \end{gather*}
\end{example}

\begin{example}[$l^2$]
    $l^2 = \seq{x: x = \seq{x_j}_{j=1}^\infty, \norm{x} = \sqrt{\sum^\infty_{j=1} \abs{x_j}^2} < +\infty}$. $(x,y) = \sum^\infty_{j=1} x_j \overline{y_j}$.
    $l^2$ -- гильбертово 
\end{example}
Главый пример
\begin{example}
    $(X, U, \mu)$ -- пространство с мерой. $L^2(X, \mu),$
     \[\norm{f} = \left( \int_X \abs{f(x)}^2 d\mu \right)^{\frac{1}{2}} < +\infty \]
    $(f,g) = \int_X f(x) \cdot \overline{g(x)} d\mu, L^2(X,\mu)$ -- полное, $\Rightarrow$ гильберттово
\end{example}

\begin{example}[пространство Харди]
    $H^2$ 2 = пространство Харди
    \[ H^2 = \seq{f(z) = \sum^{+\infty}_{n=0} a_n z^n, \norm{f}^2 = \sum^{+\infty}_{n=0} \abs{a_n}^2 < +\infty} \]
    $H^2$ линейно изомеметрически изоморфно $l^2$. 
    \[ (f,g) = \sum^\infty_{n=0} a_n \overline{b_n}, g(z) = \sum^{+\infty}_{n=0} b_n z^n \Rightarrow H^2 \text{ гильбертово} \] 
\end{example}

Отметим, где $f$ будет аналитической

\begin{gather*}
    \sum^{+\infty{n=0}} \abs{a_n}^2 < + \infty \Rightarrow \liml_{n \to \infty} \abs{a_n} = 0 \Rightarrow \overline{\liml_{n \to \infty}} \sqrt[n]{\abs{a_n}} \leq 1 \\
    \Rightarrow R \geq 1
    \intertext{где  R -- радиус круга сходимости ряда  $\sum^{+\infty}_{n=0} a_n z^n$}
    R = \frac{1}{\overline{\liml_{n \to \infty}} \sqrt[n]{\abs{a_n}}}, f \in H^2 \Rightarrow f \text{ аналитическая в } \seq{z: \abs{z} < 1}
\end{gather*}

 Теперь примеры предгильбертовых пространств
 \begin{example}
    $F$ -- финитные последовательности. \\
    $(x,y) \in F, (x,y) = \sum^\infty_{j=1} x_j \overline{y_j}$ (конечная сумма $F \subset l^2, \norm{x} = \sqrt{\sum^N_{j=1} \abs{x_j}^2}), x_{N+k} = 0 \: k \in \bN$.
    $F$ -- предгильбертово (не полное)
 \end{example}

 \begin{example}
    $C[a,b] = \seq{f: [a,b] \rightarrow \bC}$
    \[ \norm{f} = \left( \int^b_a \abs{f(x)}^2 dx \right)^{\frac{1}{2}}, (f,g) = \int^b_a f(x) \overline{g(x)} dx \]
    не полное $\Rightarrow$ предгильбертово
 \end{example}

 \begin{example}
    $\Rho - \seq{p(x) = \sum^n_{k=0} a_k x^k, a_k \in \bC, n \geq 0}$. \\
    $q(x) = \sum^n_{k=0} b_k x^k, (p,q) = \sum^n_{k=0} a_k \overline{b_k}$ предгильбертово. $\Rho$ -- линейно изометрически изоморфно $F$ : $p(x) \rightarrow (a_0, a_1, \ldots, a_n) \in F$. 
    Пополное $p$ по этой норме до гильбертова простарнства есть $l^2$.
 \end{example}

 \begin{example}
    $\Rho, \Rho \subset C[a,b]$. $(p,q) = \int^b_a p(x) \overline{q(x)} dx$ -- предгильбертово, пополнением $\Rho$ будет $L^2(a,b)$ по мере Лебега.
 \end{example}

 \begin{definition}
    $H$ -- гильбертово,
    \begin{enumerate}
        \item $x,y \in H, (x,y) = 0$, то $x \perp y $ ($x$ ортогонален $y$)
        \item $M \subset H, M$ -- подмножество. Ортогональным дополнением к нему будем называть \\
        \[ M^\perp  = \seq{y \in H: (y,x) = 0 \: \forall x \in M }\]
    \end{enumerate}
 \end{definition}

\begin{property}
    $M \subset H$ -- гильбертово $\Rightarrow M^\perp$ -- замкнутое подпространство
\end{property}

\begin{proof}
    \begin{gather*}
        y, z \in M^\perp, \lambda \in \bC, \text{ пусть } x \in M \\
        (\lambda y + z, x) = \lambda \underbrace{(y,x)}_{=0} + \underbrace{(z,x)}{=0} \Rightarrow \lambda y + z \in M^\perp \\
        \text{пусть } \seq{y_n}^\infty_{n=1}, y_n \in M^\perp, \liml_{n \to \infty} y_n = y_0, \text{ пусть } x \in M \\
        \liml_{n \to \infty} \underbrace{(y_n,x)}_{=0} = (y_0, x) \Rightarrow (y_0,x) = 0 \Rightarrow y_0 \in M^\perp
    \end{gather*}
\end{proof}

В гильбертовом пространстве всегда существует элемент наилучшего приближения, он ещё и единственный!

\begin{theorem}[о существовании элемента наилучшего приближения в гильбертовом пространстве]
    $H$ -- гильбертово, $M \subset H, M$ -- замкнутое подпространство, $x \in H \Rightarrow \: \existu y \in M : \norm{x-y} = \min_{h \in M} \norm{x-h}$
\end{theorem}

Для произвольного метрического пространства мы доказывали, что если есть конечномерное подпрострнство, то элемент существует. Доказательство начнём с простой леммы.

\begin{lemma}
    $H$ -- гильбертово, замкнутое подпространство $M \subset H$. $x \in H \setminus M, \: u,v \in M, \: d = \inf_{h \in M} \norm{x-h} $
    \[ \Rightarrow \norm{u-v}^2 \leq 2(\norm{u-x}^2 + \norm{v-x}^2) - 4d^2 \]
\end{lemma}
\begin{proof}
    Применим тождество параллелограмма к $(u-x), (v-x)$
    \[ \norm{u-v}^2 + \norm{u+v-2x}^2 = 2(\norm{u-x}^2 + \norm{v-x}^2) \]
    тут 3 слагаемых из 4 участвуютв формулировке леммы, нужно оценить только второе слагаемое. 
    \begin{gather*}
        \norm{2x - u -v} = 2 \norm{x - \frac{u+v}{2}} \geq 2d \\
        \frac{u-v}{2} \in M \Rightarrow \norm{u-v}^2 \leq 2(\norm{u-x}^2 + \norm{v-x}^2) - 4d^2
    \end{gather*}
\end{proof}
продолжение следует
\end{document}