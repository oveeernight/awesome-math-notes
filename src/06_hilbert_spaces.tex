% !TeX root = ./document.tex
\documentclass[document]{subfiles}
\begin{document}
\part{Гильбертовы пространства}
\chapter{Гильбертовы пространства}
\section{Введение}
Кто-то говорил, что матобесам в курсе ФА надо читать только гильбертовы пространства. Но неизвестно, как жить без трех китов функционального анализа, которые нас ждут дальше :(. 
А вы бы хотели 32 лекции про гильбертовы пространства?

\begin{definition}
    $H$ --- линейное пространство над $\bC$. Скалярное произведение $H \times H \rightarrow \bC$, $x,y \in H, (x,y)$ --- скалярное произведение удовлетворяет следующим аксиомам
    \begin{enumerate}
        \item $(\lambda x, y) = \lambda(x,y), \lambda \in \bC, \: x,y \in H$ 
        \item $(x, y+z) = (x,y) + (x,z)$
        \item $(y,x) = \overline{(x,y)}$ (комплексное сопряжение) 
        \item $(x,x) \geq 0, (x,x) = 0 \Leftrightarrow x = 0$
    \end{enumerate}
    если $H$ над $\bR$, то 3 выглядит как $(y,x) = (x,y)$
\end{definition}

Снабдим $H$ нормой: $\norm{x} := \sqrt{(x,x)}$ --- норма, порожденная скалярным произведением. $(H, \norm{x})$ называется предгильбертовым пространством.

Если $(H, \norm{\cdot})$ полное, то $H$ --- гильбертово.

\begin{figure}
    \centering
    \includegraphics*{images/chapter5/parallelogram.pdf}\caption{Тождество параллелограмма}
\end{figure}

\begin{property}[скалярное произведение]
    \begin{enumerate}
        \item $x,y \in H \Rightarrow |(x,y)| \leq \norm{x} \cdot \norm{y}$ (неравенство К-Б)
        \item $\norm{x} = \sqrt{(x,x)}$ удовлетворяет аксиомам нормы
        \item $\norm{x+y}^2 + \norm{x-y}^2 = 2(\norm{x}^2 + \norm{y}^2)$ (тождество параллеллограмма)
        \item непрерывность $(x,y)$, то есть $\liml_{n \to \infty} x_n = x, \liml_{n \to \infty} y_n = y \Rightarrow \liml_{n \to \infty} (x_n, y_n) = (x,y)$
    \end{enumerate}
\end{property}

\begin{proof}[2]
    \begin{gather*}
        \norm{x} = 0 \Leftrightarrow (x,x) = 0 \Leftrightarrow x = 0 \\
        \norm{\lambda x}^2 = (\lambda x, \lambda x) = \lambda \cdot \overline{\lambda} (x,x) = |\lambda|^2 \norm{x}^2
    \end{gather*}
    \begin{multline*}
        \norm{x+y}^2 = (x+y, x+y) = \norm{x}^2 + (x,y) + (y,x) + \norm{y}^2 = \\
        = \norm{x}^2 + 2 \Real (x,y) + \norm{y}^2 \leq \norm{x}^2 + 2 \norm{x} \cdot \norm{y} + \norm{y}^2 = \\
        = (\norm{x} + \norm{y})^2
    \end{multline*}
\end{proof}
Кто не верит в тождество параллелограмма, может проверить сам
\begin{proof}[4]
    \begin{multline*}
        |(x,y) - (x_n, y_n)| \leq |(x,y) - (x, y_n)| + |(x,y_n) - (x_n, y_n)| = \\
        = |(x,y - y_n)| + |(x-x_n, y_n)| \stackrel{\text{К-Б}}{\leq} \\
        \leq \norm{x} \cdot \underbrace{\norm{y-y_n}}_{\to 0} + \underbrace{\norm{x - x_n}}_{\to 0} \underbrace{\norm{y_n}}_{\leq M} \underset{n \to \infty}{\longrightarrow} 0
    \end{multline*}
    $\liml_{n \to \infty} \norm{y_n} = \norm{y} \Rightarrow \: \exists M : \norm{y_n} \leq M $
\end{proof}

\begin{example}
    \begin{gather*}
        l^2_n = \seq{x: x = \seq{x_1, \ldots, x_n}, x_j \in \bC}, \norm{x}_2 = \sqrt{\sum^n_{k=1} \abs{x_j}^2} \\
        (x,y) = \sum^n_{j=1} x_j \overline{y_j}, l^2_n \text{ --- гильбертово} \\
        y = (y_1, \ldots, y_n), y_j \in \bC, \overline{y_j} \text{ --- комплексное сопряжение}
    \end{gather*}
\end{example}

\begin{example}[$l^2$]
    $l^2 = \seq{x: x = \seq{x_j}_{j=1}^\infty, \norm{x} = \sqrt{\sum^\infty_{j=1} \abs{x_j}^2} < +\infty}$. $(x,y) = \sum^\infty_{j=1} x_j \overline{y_j}$.
    $l^2$ --- гильбертово 
\end{example}
Главый пример
\begin{example}
    $(X, U, \mu)$ --- пространство с мерой. $L^2(X, \mu),$
     \[\norm{f} = \left( \int_X \abs{f(x)}^2 d\mu \right)^{\frac{1}{2}} < +\infty \]
    $(f,g) = \int_X f(x) \cdot \overline{g(x)} d\mu, L^2(X,\mu)$ --- полное, $\Rightarrow$ гильбертово
\end{example}

\begin{example}[пространство Харди]
    $H^2$  --- пространство Харди
    \[ H^2 = \seq{f(z) = \sum^{+\infty}_{n=0} a_n z^n, \norm{f}^2 = \sum^{+\infty}_{n=0} \abs{a_n}^2 < +\infty} \]
    $H^2$ линейно изометрически изоморфно $l^2$. 
    \[ (f,g) = \sum^\infty_{n=0} a_n \overline{b_n}, g(z) = \sum^{+\infty}_{n=0} b_n z^n \Rightarrow H^2 \text{ гильбертово} \] 
\end{example}

Отметим, где $f$ будет аналитической

\begin{gather*}
    \sum^{+\infty}_{n=0} \abs{a_n}^2 < + \infty \Rightarrow \liml_{n \to \infty} \abs{a_n} = 0 \Rightarrow \overline{\liml_{n \to \infty}} \sqrt[n]{\abs{a_n}} \leq 1 \\
    \Rightarrow R \geq 1
    \intertext{где  R --- радиус круга сходимости ряда  $\sum^{+\infty}_{n=0} a_n z^n$}
    R = \frac{1}{\overline{\liml_{n \to \infty}} \sqrt[n]{\abs{a_n}}}, f \in H^2 \Rightarrow f \text{ аналитическая в } \seq{z: \abs{z} < 1}
\end{gather*}

 Теперь примеры предгильбертовых пространств
 \begin{example}
    $F$ --- финитные последовательности. \\
    $(x,y) \in F, (x,y) = \sum^\infty_{j=1} x_j \overline{y_j}$ (конечная сумма $F \subset l^2, \norm{x} = \sqrt{\sum^N_{j=1} \abs{x_j}^2}), x_{N+k} = 0 \: k \in \bN$.
    $F$ --- предгильбертово (не полное)
 \end{example}

 \begin{example}
    $C[a,b] = \seq{f: [a,b] \rightarrow \bC}$
    \[ \norm{f} = \left( \int^b_a \abs{f(x)}^2 dx \right)^{\frac{1}{2}}, (f,g) = \int^b_a f(x) \overline{g(x)} dx \]
    не полное $\Rightarrow$ предгильбертово
 \end{example}

 \begin{example}
    $\Rho = \seq{p(x) = \sum^n_{k=0} a_k x^k, a_k \in \bC, n \geq 0}$. \\
    $q(x) = \sum^n_{k=0} b_k x^k, (p,q) = \sum^n_{k=0} a_k \overline{b_k}$ предгильбертово. $\Rho$ --- линейно изометрически изоморфно $F$ : $p(x) \rightarrow (a_0, a_1, \ldots, a_n) \in F$. 
    Пополнение $p$ по этой норме до гильбертова пространства есть $l^2$.
 \end{example}

 \begin{example}
    $\Rho, \Rho \subset C[a,b]$. $(p,q) = \int^b_a p(x) \overline{q(x)} dx$ --- предгильбертово, пополнением $\Rho$ будет $L^2(a,b)$ по мере Лебега.
 \end{example}

 \begin{definition}
    $H$ --- гильбертово,
    \begin{enumerate}
        \item $x,y \in H, (x,y) = 0$, то $x \perp y $ ($x$ ортогонален $y$)
        \item $M \subset H, M$ --- подмножество. Ортогональным дополнением к нему будем называть \\
        \[ M^\perp  = \seq{y \in H: (y,x) = 0 \: \forall x \in M }\]
    \end{enumerate}
 \end{definition}

\begin{property}
    $M \subset H$ --- гильбертово $\Rightarrow M^\perp$ --- замкнутое подпространство
\end{property}

\begin{proof}
    \begin{gather*}
        y, z \in M^\perp, \lambda \in \bC, \text{ пусть } x \in M \\
        (\lambda y + z, x) = \lambda \underbrace{(y,x)}_{=0} + \underbrace{(z,x)}_{=0} \Rightarrow \lambda y + z \in M^\perp \\
        \text{пусть } \seq{y_n}^\infty_{n=1}, y_n \in M^\perp, \liml_{n \to \infty} y_n = y_0, \text{ пусть } x \in M \\
        \liml_{n \to \infty} \underbrace{(y_n,x)}_{=0} = (y_0, x) \Rightarrow (y_0,x) = 0 \Rightarrow y_0 \in M^\perp
    \end{gather*}
\end{proof}

В гильбертовом пространстве всегда существует элемент наилучшего приближения, он ещё и единственный!

\begin{theorem}[о существовании элемента наилучшего приближения в гильбертовом пространстве]
    $H$ --- гильбертово, $M \subset H, M$ --- замкнутое подпространство, $\forall x \in H \Rightarrow \: \exists! z \in M : \norm{x-z} = \min_{h \in M} \norm{x-h} = \rho(x,M)$
\end{theorem}

Для произвольного метрического пространства мы доказывали, что если есть конечномерное подпространство, то элемент существует. Доказательство начнём с простой леммы.

\begin{lemma}
    $H$ --- гильбертово, замкнутое подпространство $M \subset H$. $x \in H \setminus M, \: u,v \in M, \: d = \inf_{h \in M} \norm{x-h} $
    \[ \Rightarrow \norm{u-v}^2 \leq 2(\norm{u-x}^2 + \norm{v-x}^2) - 4d^2 \]
\end{lemma}
\begin{proof}
    Применим тождество параллелограмма к $(u-x), (v-x)$
    \[ \norm{u-v}^2 + \norm{u+v-2x}^2 = 2(\norm{u-x}^2 + \norm{v-x}^2) \]
    тут 3 слагаемых из 4 участвуютв формулировке леммы, нужно оценить только второе слагаемое. 
    \begin{gather*}
        \norm{2x - u -v} = 2 \norm{x - \frac{u+v}{2}} \geq 2d \\
        \frac{u-v}{2} \in M \Rightarrow \norm{u-v}^2 \leq 2(\norm{u-x}^2 + \norm{v-x}^2) - 4d^2
    \end{gather*}
\end{proof}

\begin{proof}
    Обозначим $d = \rho(x,M)$. Мы ещё не знаем, достигается ли расстояние, но знаем, что $\exists \seq{y_n}^\infty_{n=1}, y_n \in M$. $\liml_{n \to \infty} \norm{x-y_n} = d$. План такой: мы докажем, что последовательность фундаментальная, значит, предел
    лежит в $M$ и всё доказано.
    \begin{gather*}
        \intertext{воспользуемся леммой и устремим в получившемся неравенстве $n,m$ к $\infty$}
        \norm{y_n - y_m}^2 \stackrel{\text{лемма}}{\leq} 2( \underbrace{\norm{x-y_n}^2}_{d^2} + \underbrace{\norm{x - y_m}^2}_{d^2}) - 4d^2 \underset{n, m \to \infty}{\longrightarrow} 0 \\
        \Rightarrow \seq{y_n}^\infty_{n=1} \text{ --- фундаментальная, } H \text{ --- гильбертово } \Rightarrow \\
        \exists \liml_{n \to \infty} y_n = z, z \in M, \text{ т.к. } M \text{ замкнуто } \Rightarrow \\
        d = \liml_{n \to \infty} \norm{x - y_n} = \norm{x-z}
        \intertext{теперь проверим единственность} 
        \text{пусть } \norm{x-z} = d, \norm{x - u} = d \quad z, u \in M \\
        \intertext{воспользуемся ещё раз леммой}
        \Rightarrow \norm{z-u}^2 \leq 2(\underbrace{\norm{x-z}^2}_{=d^2} + \underbrace{\norm{x-u}^2}_{=d^2}) - 4d^2 = 0 \Rightarrow z = u
    \end{gather*}
\end{proof}

\begin{theorem}[о проекции на подпространство]
    $H$ --- гильбертово, $M \subset H$, $M$ --- замкнутое подпространство 
    \[ \forall x \in X \: \exists! z,w : x = z + w, z \in M, w \in M^\perp \]
\end{theorem}

Этот элемент $z$ как раз будет ближайшим элементом, который появился в предыдущей теореме. 

\begin{proof}
    \begin{gather*}
        d \coloneqq \rho(x,M) \quad \exists z \in M \quad \norm{x-z} = d \quad w \coloneqq x-z
        \intertext{проверим, что $w \perp M$; будем пользоваться тем, что для любой точки расстояние до $M$ больше или равно $d$}
        \text{пусть } u \in M, u \ne 0 \: \forall t \in \bR \: z + tu \in M \\
        d^2 \leq \norm{x-(z+tu)}^2 = \norm{w-tu}^2 = (w-tu,w-tu) = \underbrace{\norm{w}^2}_{=d^2} - t(u,w) - t(w,u) + t^2 \norm{u}^2 \Rightarrow
        \intertext{так как 2 и 3 слагамое комплексно сопряжённые}
        t \cdot 2 \Real(u,w) \leq t^2 \norm{u^2}
        \intertext{неравенство верно для любого вещественного t}
        \left. \begin{matrix}
            \text{пусть } t > 0 \Rightarrow 2 \Real(u,w) \leq t \norm{u}^2 \: \forall t > 0 \Rightarrow \Real(u,w) \leq 0 \\
            \text{пусть } t < 0 \Rightarrow 2 \Real(u,w) \geq t \norm{u}^2 \: \forall t < 0 \Rightarrow \Real(u,w) \geq 0
        \end{matrix} \right\} \Rightarrow \Real(u,w) = 0 \\
        \text{ аналогично } \forall t \in \bR \: d^2 \leq \norm{x - (z+itu)}^2 \Rightarrow \Imm(u,w) = 0 \\
        \Rightarrow (u,w) = 0, \text{ то есть } w \perp M \Rightarrow w \in M^\perp \\
        \intertext{осталось проверить единственность}
        \text{пусть } x = z + w, x = z_1 + w_1 \quad z, z_1 \in M, \: w, w_1 \in M^\perp \\
        \Rightarrow u = \underbrace{z-z_1}_{\in M} = \underbrace{w_1-w}_{\in M^\perp} \Rightarrow u \perp \Rightarrow (u,u) = 0 \\
        \Rightarrow u = 0 \Rightarrow z=z_1, w = w_1
    \end{gather*}
\end{proof}

\begin{definition}
    $H$ --- гильбертово, $X,Y$ --- замкнутые подпространства. $H = X \oplus Y$. $H$ --- ортогональная сумма подпространств $X$ и $Y$, если 
    \begin{enumerate}
        \item $\forall h \in H \: \exists x \in X, y \in Y : h = x + y$
        \item $\forall x \in X, y \in Y \: (x,y) = 0$
    \end{enumerate}
\end{definition}

\begin{remark}
    \item $X,Y$ --- подпространства в $H$, $X \perp Y$, то есть $\forall x \in X, \forall y \in Y \: (x,y) = 0 \Rightarrow X \cap Y = \{ 0 \}$.
\end{remark}
\begin{proof}
    $u \in X \cap Y \Rightarrow u \perp u \Rightarrow u = 0$
\end{proof}
\begin{remark}
    Если $H = X \oplus Y$, то $\forall x \in H  \: \exists! x \in X, \: \exists! y \in Y$ т.ч. $h = x + y$
\end{remark}
\begin{proof}
    Пусть $h = x + y, h = x_1 + y_1 \: x, x_1 \in X, \: y, y_1 \in Y \Rightarrow \underbrace{x - x_1}_{\in X} = \underbrace{y_1-y}_{\in Y} \stackrel{\text{Зам.1}}{\Rightarrow} x = x_1, y \in y_1$ 
\end{proof}
\begin{corollary}
    \begin{enumerate}
        \item $M$ --- замкнутое подпространство $\Rightarrow H = M \oplus M^\perp$ 
        \item $M$ --- замкнутое подпрстранство $\Rightarrow (M^\perp)^\perp = M $
        \item Если $H = X \oplus Y, \: X, Y$ --- замкнутые $\Rightarrow Y = X^\perp$
    \end{enumerate}
\end{corollary}

\begin{definition}[оператор ортогонального проектирования]
    $H$ --- гильбертово, $M$ --- замкнутое подпространство. Знаем, что $\forall x \in H \: \exists! z \in M, w \in M^\perp : h = z + w$
    \[ P_M(h) \coloneqq z \]
    $P_M$ --- оператор ортогонального проектирования на $M$. 
\end{definition}

Хоть в определении об этом нигде не сказано, но хорошо помнить, что $\norm{h-z} = \min_{y \in M} \norm{h-y}$.
На экзамене часто пристают с вопросом, откуда же взять этот $z$. $w = P_{M^\perp}(h)$.

\begin{theorem}[критерий принадлежности оператора множеству ортогональных проекторов]
    Теорема будет состоять из 2 частей. Первая полегче, в ней опишем простые свойства ортогонального проектора. Вторая посложнее, и в ней будет собственно критерий.
    \begin{enumerate}
        \item $M$ --- замкнутое подпространство, $P \coloneqq P_M \Rightarrow$ 
        \begin{enumerate}
            \item $P \in \B(H)$
            \item $P^2 = P$
            \item $(Px,y) = (x,Py), \: \forall x,y \in  H$ (по секрету, это самосопряжённость) 
        \end{enumerate}
        \item пусть оператор $P$ удовлетворяет свойствам 1-3 $\Rightarrow M \coloneqq P(H), M$ --- замкнутое, $P = P_M$
    \end{enumerate}
\end{theorem}

\begin{proof}[1 часть]
    \begin{enumerate}
        \item Сначала проверим, что $P_M \in \Lin(H,M)$
    \begin{gather*}
        h \in H \Rightarrow \: \exists! z \in M, w \in M^\perp \: h = z + w
        \intertext{утверждается, что $P(h) = z$}
        \alpha \in \bC \Rightarrow \alpha h = \alpha z + \alpha w \quad \alpha z \in M, \alpha w \in M^\perp
        \intertext{по единственности разложения $ \alpha z \Rightarrow$}
        P(\alpha h) = \alpha z \\
        \text{пусть } h_1 \in H \Rightarrow h_1 \in z_1 + w_1 \: z_1 \in M, w_1 \in M^\perp \\
        P(h_1) = z_1 \Rightarrow h+h_1 = \underbrace{(z+z_1) + (w+w_1)}_{\text{разложение единственно}} \: z + z_1 \in M, w + w_1 \in M^\perp \\
        \Rightarrow P(h + h_1) = z+z_1 = P(h) + P(h_1)
    \end{gather*}
    Теперь проверим непрерывность $P$
    \begin{gather*}
        h = z + w, \: z \perp w \Rightarrow (h,h) = (z,z) + (w,w) \\
        \norm{h}^2 = \norm{z^2} + \norm{w}^2 \\
        z = P(h) \Rightarrow \norm{P(h)}^2 \leq \norm{h}^2 \Rightarrow P \in \B(H) \\
        \norm{P} \leq 1 \\
        \text{если } M \ne \{ 0 \}, \: \exists x \in M, x \ne 0 \Rightarrow Px = x \Rightarrow \norm{P} \geq \frac{\norm{Px}}{\norm{x}} = 1 \\
        \Rightarrow \norm{P} = 1
    \end{gather*}
    \item $x \in M \Rightarrow Px = x$, 
        \[ \text{пусть } y = Px \Rightarrow y \in M \Rightarrow \underbrace{Py}_{=y=Px} = P(Px) \Rightarrow P^2x = Px \] 
    \item
    \begin{gather*}
         x, y \in H, P = P_m, Q = P_{M^\perp} \\
         x = Px + Qx, y = Py + Qy \\
         (Px,y) = (Px, Py + Qy) = (Px, Py) \\
         (x, Py) = (Px + Qx, Py) = (Px, Py) 
    \end{gather*}
\end{enumerate}
\end{proof}

\begin{proof}[2 часть]
    $p \in \B(H), M \coloneqq P(H), M$ --- подпространство в алгебраическом смысле.
    План такой: проверим, что $P$ совпадает с ортогоналным проектором на $M$ и что он отправляет ортогональное дополнение в $0$.
    Проверим, что если $x \in M$, то $Px = x$. 
    \begin{gather*}
        \left.
        \begin{matrix}
            \text{пусть } x \in M \Rightarrow \: \exists y \in H : Py = x \Rightarrow P(Py) = Px \\
            \text{по свойству ортогонального оператора }  P^2 = P \Rightarrow P(Py) = Py = x 
        \end{matrix} \right\} \Rightarrow \\
        \Rightarrow x = Px
        \intertext{Проверим теперь замкнутость $M$}
        \text{пусть } \seq{x_n}^\infty_{n=1}, x_n \in M, \liml_{n \to \infty} x_n = x_0 \Rightarrow \liml_{n \to \infty} P x_n = P x_0  \\
        P x_n = x_n \Rightarrow \liml_{n \to \infty} x_n = P x_0 \Rightarrow x_0 = P x_0 \Rightarrow x_0 \in P(H) = M 
        \intertext{осталось убедиться, что оператор $P$ отправляет в $0$ ортогональное дополнение}
        \text{пусть } y \in M^\perp \\
        \norm{Py}^2 = (Py, Py) \stackrel{\text{самосопряжённость}}{=} (y, P(Py)) = (y, Py) \text{т.к. } y \in M^\perp, Py \in M = 0 \\
        \Rightarrow Py = 0
    \end{gather*}
\end{proof}

Мы знаем, что оператор совпадает на $M$, а ортогональное дополнение отправляет в $0$
\begin{gather*}
    h \in H \Rightarrow h = z + w, z \in M, w \in M^\perp \\
    \Rightarrow P(z+w) = z \\
    P_m(z+w) = z \\
    \Rightarrow P = P_m
\end{gather*}

\begin{corollary}[ортогональный оператор на конечномерное подпространство]
    $H$ ---  гильбертово, подпространство $M \subset H, \dim M = n, n \in \bN$
    \begin{gather*}
        \seq{e_j}^n_{j=1} \text{ --- ортонормированный базис} \\
        (e_j, e_k) = \begin{cases}
            0 & j \ne k \\
            1 & j = k
        \end{cases},
        x \in H, P_M(x) = \sum^n_{j=1} (x, e_j) e_j
    \end{gather*}
\end{corollary}

\begin{proof}
    $s_n = \sum^n_{j=1} (x,e_j) e_j, \: s_n \in M, w \coloneqq x - s_n$.
    Проверим, что $w \in M^\perp$. Для этого проверим, что он ортогонален всем $e_j$
    \begin{gather*}
        (s_n, e_k) = \left( \sum^n_{j=1} (x, e_j) e_j, e_k \right) = (x, e_k) \\
        \Rightarrow (x-s_n, e_k) = 0 \Rightarrow (w, e_k) = 0 \: \forall k, 1 \leq k \leq n \\
        \Rightarrow w \perp M, \Rightarrow w \in M^\perp \Rightarrow P_M(x) = s_n
    \end{gather*}
\end{proof}

\begin{corollary}[критерий полноты системы элементов в гильбертовом пространстве]
    $H$ --- гильбертово, $\seq{x_\alpha}_{\alpha \in A}, x_{\alpha} \in H$ ($A$ --- множество индексов)
    \[ \seq{x_\alpha}_{\alpha \in A} \text{ --- полное } \Leftrightarrow (y \perp x_{\alpha} \: \forall \alpha \in A \Rightarrow y = 0) \]
\end{corollary}

\begin{proof}
    \begin{gather*}
        \seq{x_\alpha}_{\alpha \in A} \text{ --- полное } \Rightarrow \overline{\calL \seq{x_\alpha}_{\alpha \in A}} = H \\
        L = \overline{\calL \seq{x_\alpha}} \\
        L = H \Leftrightarrow L^\perp = \{0 \} \Leftrightarrow (y \perp x_\alpha \: \forall \alpha \in A \Rightarrow y = 0)
    \end{gather*}
\end{proof}
Несмотря на то, что доказательство тривиальное, этот критерий полноты очень полезен. \\

Упражнения, которое когда-то давали в качестве задачи на 5 на экзамене
\begin{statement}
    $l^2, L = \seq{x = \seq{x_n}^\infty_{n=1} \in l^2: \sum^\infty_{n=1} x_n = 0}$. Нужно доказать, что $L$ --- плотно в $l^2$
\end{statement}

\begin{statement}
    $z \in \bC, \abs{z} < 1, x_z = \{1, z, z^2, \ldots, z^n, \ldots \} \in l^2$. $\seq{z_n}^\infty_{n=1}, \abs{z_n} < 1, \liml_{n \to \infty} z_n$
    Нужно доказать, что $\seq{x_{z_n}}^\infty_{n=1}$ --- плотное семейство в $l^2$
\end{statement}

\begin{statement}
    Пусть $\liml_{n \to \infty} z_n = a, \abs{a} < 1$. Нужно доказать, что $\seq{x_{z_n}}^\infty_{n=1}$ --- плотное семейство в $l^2$
\end{statement}

То, что $\abs{a} < 1$ --- очень важно. При равенстве утверждения неверны.

\begin{definition}[коэффициент Фурье]
    $H$ --- гильбертово, $\seq{e_n}^\infty_{n=1}$ --- ортонормированная система
    \begin{gather*}
        (e_j, e_k) = 0 \text{ при } j \ne k \\
        (e_k, e_k) = 1, \norm{e_k} = 1 \\
        M_n = \seq{ \alpha e_n | \alpha \in \bC}, \dim M_n = 1, P_{M_n} \\
        x \in H, P_{M_n}(x) = (x, e_n) e_n \\
        (x, e_n) \text{ --- коэффициент Фурье} \\
        x \sim \sum^\infty_{n=1} (x, e_n) e_n \text{ ряд Фурье по системе } \seq{e_n}^\infty_{n=1} 
    \end{gather*}
\end{definition}

\begin{definition}
    \begin{gather*}
        \seq{e_n}^\infty_{n=1} \text{ --- ортогональная система  (ОС)} \\
        (e_j, e_k) = 0, j \ne k, e_n \ne 0 \\
        M_n = \seq{\alpha e_n: \alpha \in \bC} \\
        P_{M_n}(x) = \left(x, \frac{e_n}{\norm{e_n}} \right), \frac{e_n}{\norm{e_n}} = \frac{(x,e_n)}{\norm{e_n}^2} e_n \\
        \text{коэффициент Фурье по системе} \seq{e_n} \\
        x \sim \sum^\infty_{n=1} \frac{\norm{(x,e_n)}}{{\norm{e_n}^2}} e_n
    \end{gather*}
\end{definition}

Когда мы пишем $\seq{e_n}^\infty_{n=1}$, мы подразумеваем бесконечномерность пространства. Если же вы возьмёте книжку Колмогорова, то гильбертово пространство в ней по определению бесконечномерное. Однако И.В. решил убрать это условие в своём курсе,
Ввдь есть теория конечномерных банаховых пространств, где переходят к пределу
и получают утверждения про бесконечномерные пространства. В общем: если вам попадётся кровожадный помощник на экзамене и вы скажете, что гильбертово пространство бесконечномерное, он спросит: <<С какоq стати?>>. Если не скажете --- то он скажет, что вы даже не знаете определение,
и вы в любом случае получите 2.
\begin{corollary}[неравенство Бесселя]
    $H$ --- гильбертово, $\seq{e_n}^\infty_{n=1}$ --- О.Н.С, $x \in H \Rightarrow$
    \[ \sum^\infty_{n=1} \abs{(x,e_n)}^2 \leq \norm{x}^2 \] 
\end{corollary}
\begin{proof}
    \begin{gather*}
        h = \sum^n_{j=1} \alpha_j e_j, \alpha_j \in \bC \Rightarrow \\
        \norm{h}^2 = \left( \sum^n_{j=1} \alpha_j e_j, \sum^n_{k=1} \alpha_k e_k \right) = \sum^n_{k=1} \abs{\alpha_k}^2 \\
        L_n = \calL \seq{e_j}^n_{j=1}, P_{L_n}(x) = \sum^n_{j=1} (x, e_j) e_j \\
        \norm{P_{L_n}} \leq 1 \Rightarrow \norm{P_{L_n}(x)}^2 \leq \norm{x}^2 \Rightarrow \\
        \sum^n_{j=1} \abs{(x,e_j)} \leq \norm{x}^2 \: \forall n \in \bN \Rightarrow \\
        \sum^\infty_{j=1} \abs{(x,e_j)}^2 \leq \norm{x}^2
    \end{gather*}
\end{proof}

Cейчас выясним, когда неравенство превращается в равенство, то есть когда можно узнать норму, вычислив эту сумму.


\begin{theorem}[о разложении элемента гильбертова пространства в ряд Фурье]
    $H$ --- гильбертово, $x \in H, \seq{e_n}^\infty_{n=1}$ --- О.Н.С., тогда следующие условия равносильны
    \begin{enumerate}
        \item $x \in \overline{\calL \seq{e_n}^\infty_{n=1}} $
        \item $x = \sum^\infty_{n=1} (x,e_n) e_n $
        \item $\norm{x}^2 = \sum^\infty_{n=1} \abs{(x,e_n)}^2$ (равенство Парсеваля)
    \end{enumerate}
\end{theorem}

\begin{proof}
    $1 \Rightarrow 2$ \\
    По виду первое утверждение куда более слабое, чем второе. В первом можно приблизить элемент сколько угодно хорошо какими-то элементами. 
    Во втором же есть сходимость к какому-то ряду. 
    \begin{gather*}
        x \in H, x \in \overline{\calL \seq{e_n}^\infty_{n=1}}, \text{ пусть } \varepsilon > 0 \\
        \exists y = \sum^n_{k=1} \alpha_k e_k, \norm{x-y} < \varepsilon \\
        L_n = \calL \seq{e_k}^n_{k=1} \Rightarrow \rho(x, L_n) < \varepsilon \quad P_{L_n}(x) = \underbrace{\sum^n_{j=1} (x,e_j) e_j}_{\coloneqq s_n} \\
        \Rightarrow \norm{x - s_n} \leq \norm{x-y} < \varepsilon \quad L_n \subset L_{n+1} \Rightarrow \\
        \norm{x - S_{n+1}} \leq \norm{x - S_n} < \varepsilon \Rightarrow \\
        \forall m \geq n \: \norm{x-S_m} < \varepsilon \Rightarrow \liml_{n \to \infty} S_n = x \\
    \end{gather*}
    $2 \Rightarrow 1$ очевидно: $x = \liml_{n \to \infty} s_n \Rightarrow x \in \overline{\calL \seq{e_j}^\infty_{j=1}}$ \\
    $2 \Rightarrow 3$ \\
    $s_n = \sum^n_{k=1} (x,e_k) e_k, \: x = \liml_{n \to \infty} s_n$, и по непрерывности скалярного произведения $\Rightarrow (x,x) = \liml_{n \to \infty} (s_n, s_n) \Leftrightarrow$
    \[ \norm{x}^2 = \liml_{n \to \infty} \sum^n_{k=1} \abs{(x, e_k)}^2 = \sum^\infty_{k=1} \abs{(x,e_k)}^2 \]
    $3 \Rightarrow 2$ \\
    \begin{gather*}
        \sigma_n = \sum^n_{k=1} \abs{(x,e_k)}^2, \: \liml_{n \to \infty} \sigma_n = \norm{x}^2 \\
        w_n \coloneqq x - s_n, \: w_n \perp s_n \Rightarrow \norm{x}^2 = \underbrace{\norm{s_n}^2}_{\underset{n \to \infty}{\longrightarrow} \norm{x}^2} + \norm{w_n}^2 \\
        \norm{s_n}^2 = \sigma_n \Rightarrow \liml_{n \to \infty} \norm{w_n}^2 = 0 \Rightarrow \liml_{n \to \infty} \norm{x - s_n} = 0
    \end{gather*}
\end{proof}

\begin{corollary}
    $H$ --- гильбертово, $\seq{e_n}^\infty_{n=1}$ --- полная О.Н.С $\Rightarrow$ 
    \[ \forall x \in H \: x = \sum^\infty_{k=1}(x,e_k) e_k, \norm{x}^2 = \sum^\infty_{k=1} \abs{(x,e_k)}^2 \]
\end{corollary}
Доказывать нечего, принадлежность линейной оболочке означает полноту.
\begin{definition}
    $(X, \norm{\cdot})$ --- нормированное пространство, $\seq{e_n}^\infty_{n=1}$ --- базис (Шаудера), если 
    \[ \forall x \in X \: \exists! \seq{\alpha_n}^\infty_{n=1}, \: \alpha_n \in \bC : x = \sum^\infty_{n=1} \alpha_n e_n \] 
\end{definition}

\begin{example}
    $l^p, 1 \leq p < + \infty, e_n = (0, 0, \ldots, 0, \underbrace{1}_{n}, 0, \ldots)$
    \begin{gather*}
        x \in l^p, x = \seq{x_n}^\infty_{n=1} = \sum^\infty_{n=1} x_n e_n, \norm{x-s_n}_{l^p} \underset{n \to \infty}{\longrightarrow} 0 \\ 
        c_0, x \in c_0, \liml_{n \to \infty} x_n = 0 \quad x = \sum^\infty_{n=1} x_n e_n \quad \norm{x - s_n}_\infty \underset{n \to \infty}{\longrightarrow} 0
    \end{gather*}
\end{example}

Упражнение: $c = \seq{x = \seq{x_n}^\infty_{n=1}, \: \exists \liml_{n \to \infty} x_n = x_0} \subset l^\infty$. Что тут будет базисом?

\begin{remark}
    Если в $(X, \norm{\cdot})$ есть базис, то $X$ --- сепарабельно.
\end{remark}
\begin{remark}[Проблема Банаха, проблема базиса]
    Проблема Банаха, проблема базиса
    \[ X \text{{ --- нормированное сепарабельное}} \stackrel{?}{\Rightarrow} \: \exists \text{ базис} \]
    Собирались товарищи во Львове в кафе и выводили эти проблемы. Обычно математики любят сидеть в тиишине, нот вот Банах любил сидеть в кафе. Вероятно, они там не только чаи гоняли.
    Пер Энфло в 1973 году дал ответ на этот вопрос: нет. Он предоставил множество контр-примеров. Да и вообще он знаменит своими контр-примерами. Сейчас в Америке где-то работает.
\end{remark}

\begin{corollary}
    $H$ --- гильбертово, $\seq{e_n}^\infty_{n=1}$ --- полная О.Н.С. $\Rightarrow \seq{e_n}^\infty_{n=1}$ --- базис в $H$
\end{corollary}

\begin{proof}
    \begin{gather*}
        x \in H \Rightarrow x = \sum^\infty_{n=1} (x, e_n) e_n \\
        \text{проверяем единственность: пусть } x = \sum^\infty_{n=1} \alpha_n e_n, \alpha_n \in \bC \\
        \sigma_n = \sum^n_{k=1} \alpha_k e_k \: \liml_{n \to \infty} \sigma_n = x \Rightarrow \liml_{ n \to \infty} (\sigma_n, e_k) = (x, e_k) \\
        \text{пусть } n \geq k \Rightarrow (\sigma_n, e_k) = \alpha_k \Rightarrow \alpha_k = (x, e_k)
    \end{gather*}
\end{proof}


\begin{theorem}[о существовании О.Н.Б. в сепарабельном гильбертовом пространстве]
    $H$ --- сепарабельное гильбертово пространство $\Rightarrow$
    \[ \exists \seq{e_n}^\infty_{n=1} \text{ --- О.Н.Б. } \]    
\end{theorem}

По секрету, если убрать сепарабельность, то базис будет несчётный. Какова размерность, такой и базис. Обычно, когда говорят о гильбертовом пространстве, подразумевают гильбертово сепарабельное.
\begin{proof}
    Будем действовать в 2 этапа. Сепарабельность означает, что есть счётное всюду плотное множество, возьмём его: $\seq{x_n}^\infty_{n=1}$.
    1 этап: по индукции выберем из нгео линейно независимую систему так, чтобы замыкание их линейной оболочки совпадало с замыканием линейной оболочки $x_n$. Оно будет полным и линейно-независимым
    Потом применим к нему ортогонализацию Грама-Шмидта (а он ученик Гильберта, кстати) \\
    \begin{gather*}
        x_1 = x_2 = \ldots = x_{n_1 - 1} = 0, x_{n_1} \ne 0 \quad z_1 = x_{n_1} \\
        L_1 = \calL(z_1) = \seq{\alpha z_1 | \alpha \in \bC } \\
        x_{n_1+1}, \ldots, x_{n_2 -1} \in L_1 \: x_{n_2} \notin L_1, z_2 = x_{n_2}, L_2 = \calL (z_1, z_2) \\
        \text{пусть выбрали } z_1, \ldots, z_{m} \\
        z_m = x_{n_m}, x_{n_m + 1}, \ldots, x_{n_{m+1}-1} \in L_m, x_{n_{m+1}} \notin L_m \\
        z_{m+1} = x_{n_{m+1}}
        \intertext{как мы их выбираем?}
        \seq{z_j}^\infty_{j=1} \text{ --- линейно независимы} \\
        \calL(z_j)^m_{j=1} = \calL \seq{x_k}^{n_m}_{k=1} \: \forall m \Rightarrow \calL \seq{z_j}^\infty_{j=1} = \calL \seq{x_n}^\infty_{n=1} \\
        \Rightarrow H = \overline{\calL \seq{z_n}^\infty_{n=1}} \Rightarrow \seq{z_n}^\infty_{n=1} \text{ --- полная}
    \end{gather*}
        Процесс ортогонализации Грама-Шмидта:
        \begin{gather*}
            e_1 = \frac{z_1}{\norm{z_1}}, \text{ пусть } e_1, \ldots, e_{n-1} \text{ --- выбрали } \calL \seq{e_j}^{n-1}_{j=1} = \calL \seq{z_j}^{n-1}_{j=1} \\
            L_n = \calL \seq{z_j}^n_{j=1}, L_n \subsetneq L_{n+1} \quad e_n = \frac{z_n - P_{L_{n-1}}(z_n)}{\norm{z_n-P_{L_{n-1}}(z_n)}} = \frac{z_n - \sum^{n=1}_{j=1} (z_n, e_j) e_j}{\norm{z_n-P_{L_{n-1}}(z_n)}} \Rightarrow \\
            \seq{e_n}^\infty_{n=1} \text{ --- полная О.Н.С.} \Rightarrow \\
            \seq{e_n}^\infty_{n=1} \text{ --- базис (Шаудера)}
        \end{gather*}
    \end{proof}
        
Теперь докажем, что все сепарабельные линейные пространства похожи друг на друга как две капли воды: не просто линейно изоморфны, а линейно изометрически изоморфно. Для конечномерных
тоже верно, нужно только рассматривать пространства одинаковой размерности.
\begin{theorem*}
    Все сепарабельные бесконечномерные гильбертовы пространства линейно изометрически изоморфны друг другу
\end{theorem*}

\begin{proof}
    $H$ ---  гильбертово сепарабельное, $\dim H = \infty$.
    Мы обсуждали, что линейный изоморфизм --- отношение эквивалентности, отношение изометричности --- тоже. Поэтому линейный изометрический изоморфизм есть отношение эквивалентности.
     Поэтому вместо того, чтобы брать $H_1, H_2$, возьмём $H$ и $l^2$ и покажем, что они линейно изометрически изоморфны.
     \begin{gather*}
        \text{пусть } \seq{f_n}^\infty_{n=1} \text{ --- О.Н.Б. в } H \\
        \varphi: H \rightarrow l^2 \quad x \in H \quad x \mapsto \seq{(x, f_n)}^\infty_{n=1} \\
        \norm{x}^2 = \sum^\infty_{n=1} \abs{(x,f_n)}^2 \Rightarrow \norm{x}_H = \norm{\varphi(x)}_{l^2} \\
        \varphi \in \Lin(H, l^2) \text{ очевидно } \Rightarrow \varphi \in \B(H,l^2) \\
        \varphi \text{ --- инъективен } 
        \intertext{проверим, что $\varphi$ --- сюръекция}
        \text{пусть } y = \seq{y_n}^\infty_{n=1} \in l^2 \\
        s_n = \sum_{k=1}^n y_k f_k, s_n \in H, \text{ пусть m > n } \\
        \norm{s_m - s_n}^2 = \sum^m_{k = n + 1} \abs{y_k}^2 \underset{n, m \to \infty}{\longrightarrow} 0 \Rightarrow \seq{s_n} \text{ --- фундаментальная} \\
        \Rightarrow \: \exists \liml_{n \to \infty} s_n = s, s = \sum^\infty_{k=1} y_k f_k \Rightarrow \varphi(s) = y
     \end{gather*}
\end{proof}

\begin{remark}
    Пусть $ m \in \bN$, $H$ --- гильбертово пространство, $\dim H = m \Rightarrow H$ --- линейно изометрически изоморфно $l^2_m$. 
\end{remark}

\section{Пространство, сопряжённое к гильбертову}


Опишем все непрерывные функционалы в гильбертовом пространстве $H$.
\begin{theorem*}[Ф.Рисс, общий вид линейного непрерывного функционала в гильбертовом пространстве]
    $H$ --- гильбертово. Опишем набор линейных функционалов: покажем, что он непрерывный. Вторая часть будет утверждать, что других нет.
    \begin{enumerate}
        \item $y \in H, y$ --- фиксирован. Рассмотрим отображение 
        \begin{gather*}
            f_y: H \rightarrow \bC \quad x \mapsto (x,y) \: \forall x \in H \\
            \Rightarrow f_y \in H^*, \norm{f_y}_{H^*} = \norm{y}_H
        \end{gather*}
        \item $ f \in H^* \Rightarrow \: \exists! y \in H : f = f_y$, то есть $f(x) = (x,y) \: \forall x \in H$
    \end{enumerate}
\end{theorem*}
\label{chap6:riss}

\begin{proof}[1 часть]
    \begin{gather*}
        f_y \in \Lin(H, \bC) \text{ --- очевидно из свойств скалярного произведения} \\
        \abs{f_y(x)} = \abs{(x,y)} \stackrel{\text{К-Б}}{\leq} \norm{x} \cdot \norm{y} \: \forall x \in  H \\
        \Rightarrow f_y \in H^*, \norm{f_y}_{H^*} \leq \norm{y}_H \\
        \intertext{проведём тривиальное отбрасывание тривиальных случаев}
        y = 0 \Rightarrow f_y = \bZero \quad \norm{f_y} = 0\\
        \text{пусть } y \ne 0 \quad \norm{f_y} = \sup_{x \in H, x \ne 0} \frac{\abs{f_y(x)}}{\norm{x}} \geq \frac{\abs{f_y(y)}}{\norm{y}} = \frac{(y,y)}{\norm{y}} = \norm{y} \\
        \Rightarrow \norm{f_y}_{H^*} = \norm{y}_H
    \end{gather*}
\end{proof}

\begin{proof}[2 часть]
    Намёк, откуда брать $y$: мы знаем, что $f_y(x) = 0 \Leftrightarrow (x,y) = 0 \Leftrightarrow x \in \seq{y}^\perp$. Сначала рассмотрим и отбросим тривиальный случай:
    пусть $f(x) = \bZero$, то есть $f(x) = 0 \: \forall x \in H \Rightarrow$ пусть $y = 0, f = f_0$. \\
    Теперь пусть $f \ne \bZero, N = \Ker f (N = f^{-1}(0)) \Rightarrow n \subsetneq H, f$ --- непрерывный $\Rightarrow N$ --- замкнутое подпространство. Значит, существует нетривиальное ортогональное дополнение $N^\perp$, то есть $N^\perp \ne \seq{0}$,
    пусть $x_0 \in N^\perp, x_0 \ne 0$
    \begin{gather*}вообще, это
        v = \frac{x_0}{f(x_0)}, f(x_0) \ne 0, f(v) = 1, f(v) = \frac{1}{f(x_0)} \cdot f(x_0) = 1
        \intertext{установим следующую вещь: $\dim N^\perp = 1$, то есть все элементы дополнения кратны $v$;
          вообще, это очевидно, гомоморфный образ группы изоморфен факторгруппе по ядру гомоморфизма, помните такую скороговорку из алгебры? но сейчас докажем аккуратно}
        \text{пусть } u \in N^\perp \quad \alpha \coloneqq f(u) \quad f(\alpha v) = \alpha f(v) = \alpha \\
        \left. \begin{matrix}
            \Rightarrow f(u - \alpha v) = 0 \Rightarrow u - \alpha v \in N \\
            u,v \in N^\perp \Rightarrow u - \alpha v \in N^\perp
        \end{matrix} \right\} u - \alpha v = 0 \Rightarrow u = \alpha  v \\
        \forall u \in N^\perp f(u) = \alpha \Rightarrow u = \alpha v \\
        u = \alpha v \Rightarrow f(u) = \alpha
        \intertext{$v$ уже почти то, что нам надо, но мы его еще должны нормировать, чтобы не отправлять те же элементы в 0, что и $f$; найдём $\beta: f_{\beta v} (v) = 1 = f(v)$}
        f_{\beta v} (v) = (v, \beta v) = \overline{\beta} \norm{v}^2 = 1 \Rightarrow \beta = \frac{1}{\norm{v}^2} \\
        y = \frac{v}{\norm{v}^2} \\
        \text{пусть } x \in H, x = h + \alpha v, h \in N, \alpha v \in N^\perp\\
        f(x) = \alpha, f_y(x) = \alpha \Rightarrow f = f_y
    \end{gather*}
    Всё, что осталось проверить, это единственность:
    \begin{gather*}
        f_y = f_z \Rightarrow (x,y) = (x,z) \: \forall x \in H \\
        \Rightarrow (x,y-z) = 0 \: \forall x \in H \Rightarrow y - z = 0 
    \end{gather*}
\end{proof}

\begin{remark}
    Рассмотрим отображение $C: H \rightarrow H^*, C(y) = f_y$. Во-первых, с суммой всё в порядке: $C(y+z) = f_{y+z} = f_y + f_z = C(y) + C(z)$. А с умножением на комплексное число уже не всё хорошо:
    пусть $\alpha \in \bC, C(\alpha y) = f_{\alpha y}, f_{\alpha y} = (x, \alpha y) = \overline{\alpha}(x,y) = \overline{\alpha} f_y(x), C(\alpha y) = \overline{\alpha} C(y)$ , то есть умножение не совсем линейное.
    Но $\norm{C(y)}_{H^*} = \norm{y}_H$, $C$ --- \textbf{антилинейный изометрический изоморфизм}.
    Удобно думать, что сопряжённое к гильбертову пространство --- это оно само. Говорят: $H^* = H$, а имеют в виду это взаимно-однозначное соответствие $C(H) = H^*$. Это очень просто, но фантастически удобно: сопряжённое ---
    это оно само, но за удобство надо платить: $\alpha$ переходит в $\overline{\alpha}$.
\end{remark}

\begin{example}
    Есть $l^2, (x,y) = \sum_{n=1}^\infty x_n \overline{y_n}, \: x,y \in l^2$. Как устроены все линейные функционалы в пространстве последовательностей $l^2$?$f \in (l^2)^* \Rightarrow \: \exists! y \in l^2: f(x) = (x,y) = \sum^\infty_{n=1} x_n \overline{y_n}$.
\end{example}
\begin{example}
    $(X, \mu)$, $L^2(X,\mu), (f,g) = \int_X f(x) \overline{g(x)} d \mu$
    \[ F \in (L^2(X,\mu))^* \Rightarrow \: \exists! g \in L^2(X\mu) : F(f) = \int_X f(x) \overline{g(x)} d \mu \]
\end{example}

Посмотрим сейчас чуть-чуть, как эта теория применяется к классическим рядам Фурье, которые были у нас в анализе.

\section{Классические ряды Фурье}
Как сходятся ряды Фурье в $L^2$ по мере Лебега?
\begin{example}
    \[ L^2_{\bR} [-\pi,\pi] \text{ по мере Лебега } dx, (f,g) = \int^\pi_{-\pi} f(x)g(x) dx, \{1, \cos nx, \sin nx\}_{n \in \bN} \]
\end{example}
Для того, чтобы что-то утверждать, нам понадобится второй вариант теоремы Вейерштрасса: но доказывать мы его не будем. 
\begin{theorem*}[Вейерштрасса]
    $f \in \tilde{C}_{\bR}[-\pi, \pi] (f \in C[-\pi, \pi], f(-\pi) = f(\pi))$
    \begin{multline*}
        \forall \varepsilon > 0 \: \exists T(x) = a_0 + \sum^n_{k=1} a_k \cos kx + b_k \sin kx \\
         \norm{f-T}_\infty = \max_{x \in [-\pi,\pi]} \abs{f(x) - T(x)} < \varepsilon
    \end{multline*}
    то есть существует многочлен, который приближает нашу функцию с точностью до $\varepsilon$
\end{theorem*}

\begin{theorem}
    $\{1, \cos nx, \sin nx\}_{n \in \bN}$ --- полная О.С. в $L^2_{\bR}[-\pi, \pi]$
\end{theorem}
\begin{proof}
    \begin{gather*}
        \int^\pi_{-\pi} \cos (nx) \sin(mx) dx = 0 \quad \int^\pi_{-\pi} (\cos (nx))^2 dx = \pi = \int^\pi_{-\pi} (\sin nx)^2 dx \\
        \int^\pi_{-\pi} 1 dx = 2\pi \quad \int^\pi_{-\pi} \cos nx \cos(mx) dx = 0 (n \ne m) \\
        \int^\pi_{-\pi} \sin nx \sin(mx) dx = 0 (n \ne m) \\
        \Rightarrow \{ 1, \cos nx, \sin nx \}_{n \in \bN} \text{ --- ортонормированная система}
        \intertext{мы уже доказали, что $C[-\pi,\pi]$ плотно в $L^2[-\pi,\pi]$ по мере Лебега, то есть любую функцию из $L^2$ можно приблизить сколь угодно хорошо,
        найдя такую функцию $g$, что разница интегралов будет меньше $\varepsilon$, но $g$  в отличие от $f$ --- $2\pi$-периодическая}
        \Rightarrow \tilde{C}[-\pi,\pi] \text{ плотно в } L^2[-\pi,\pi] \\
        \exists g \in \tilde{C}[-\pi,\pi] \quad \left( \int^\pi_{-\pi} \abs{f(x) - g(x)}^2 dx \right)^{\frac{1}{2}} < \varepsilon  % каждый из интегралов и от $f$, и от $g$ будет меньше $\frac{\varepsilon}{2} 
        \exists \delta > 0: g(x) = f(x), x \in [-\pi, \pi - \delta] \\
        \Rightarrow \tilde{C}[-\pi,\pi] \text{ плотно в } L^2[-\pi,\pi] \\
        \forall \varepsilon > 0 \: \forall f \in L^2[-\pi,\pi] \: \exists g \in \tilde{C}[-\pi,\pi], \norm{f-g}_{L^2} < \varepsilon
        \intertext{по теоремере Вейерштрасса $\exists T = \alpha_0 + \sum^n_{k=1} \alpha_k \cos kx + \beta_k \sin kx$}
        \norm{g-T}_\infty < \varepsilon \Rightarrow \norm{g-T}_2 = \left( \int^\pi_{-\pi} \abs{g(x) - T(x)}^2 dx \right)^{\frac{1}{2}} < (\varepsilon^2 \cdot 2\pi)^{\frac{1}{2}} = \sqrt{2\pi} \cdot \varepsilon \\
        \Rightarrow \norm{f-T}_2 < \varepsilon(1 + \sqrt{2\pi}) \Rightarrow \{1, \cos nx, \sin nx \} \text{ --- полная}
    \end{gather*}
\end{proof}

\begin{corollary}
    Пусть $f \in L^2_{\bR}[-\pi,\pi]$. Коэффициенты Фурье:
    \[ a_0 = \frac{1}{2\pi} \int^\pi_{-\pi} f(x) dx, a_k = \frac{1}{\pi} \int^\pi_{-\pi} f(x) \cos kx dx, b_k = \frac{1}{\pi} \int^\pi_{-\pi} f(x) \sin kxdx \]
    Теперь что же значит $f(x)$ разлагается в свой ряд Фурье?
    \begin{gather*}
        S_n(x) = a_0 + \sum^n_{k=1} (a_k \cos kx + b_k \sin kx) \Rightarrow \\
        \left( \int^\pi_{-\pi} \abs{f(x)-S_n(x)}^2 dx \right)^{\frac{1}{2}} \underset{n \to \infty}{\longrightarrow} 0 \tag{*} \\
        f(x) = a_0 + \sum^\infty_{k=1} (a_k + \cos kx + b_k \sin kx) \text{ в смысле (*)}
    \end{gather*}
\end{corollary}

\begin{example}
    $L^2_{\bC}[-\pi,\pi], f \in L^2_{\bC}[-\pi,\pi], f = u + iv$
    \[u, v \in L^2_{\bR}[-\pi,\pi] \Rightarrow \seq{\frac{1}{\sqrt{2\pi}}, \frac{\cos nx}{\sqrt{\pi}}, \frac{\sin nx}{\sqrt{\pi}}} \text{ -- ОНБ} \]
\end{example}

\begin{example}
    $L^2_{\bC}[-\pi, \pi], \seq{e^{inx}}_{n \in \bZ}$ --- полная О.С.
\end{example}

\begin{gather*}
    (f,g) = \int^\pi_{-\pi} f(x) \overline{g(x)} dx, (e^{inx}, e^{imx}) = \int^\pi_{-\pi} e^{i(n-m)x}dx = \begin{cases}
        0 & n \ne m \\
        2 \pi & n = m
    \end{cases} \\
    c_n = \frac{1}{2\pi} \int^\pi_{-\pi} f(x) e^{-inx} dx = \frac{1}{2} (a_n - i b_n), n \ne 0 \\
    c_ 0 = a_0 \\
    \sum^n_{k=-n} c_k e^{-ikx} = a_0 + \sum^n_{k=1} (a_k \cos kx + b_k \sin kx) = S_n x \\
    \norm{f - S_n}_2 \underset{n \to \infty}{\longrightarrow} 0 \Rightarrow \seq{e^{inx}}_{n \in \bZ} \text{ --- полная система}
\end{gather*}

\begin{example}
    $L^2_{\bR}[0,\pi], \seq{\cos nx}^{+\infty}_{n=0}$ --- полная О.С.
\end{example}

\begin{proof}
    \begin{gather*}
        f \in L^2_{\bR}[0,\pi], \text{ пусть } f(-x) = f(x), x \in (0, \pi] \\
        f \in L^2_{\bR}[-\pi,\pi], b_k = \frac{1}{\pi} \int^\pi_{-\pi} f(x) \sin kx dx = 0 \\
        \norm{f-S_n(f)}_{L^2[-\pi,\pi]} \underset{n \to \infty}{\longrightarrow} 0 \Rightarrow \norm{f - \left(a_0 + \sum^n_{k=1} a_k \cos kx\right)}_2 \underset{n \to \infty}{\longrightarrow} 0
    \end{gather*}
\end{proof}

Прощаемся с гильбертовыми пространствами.
\end{document}