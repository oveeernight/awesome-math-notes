% !TeX root = ./document.tex
\documentclass[document]{subfiles}
\begin{document}
\chapter{Линейные операторы}
 
Первый парагарф про линейные пространства будет совсем простой, здесь будут самые тривиальные свойства, следующие из линейности.
\section{Линейные операторы в линейных пространствах} % мб главу так назвать
 
\begin{definition}[Линейный оператор]
    $X,Y$ ~-- линейны над $k (k = \bR$ или $\bC)$. $A: X \rightarrow Y, A$ ~-- \textbf{ линейный оператор}, если 
    \[ A(\alpha x + \beta z) = \alpha A x + \beta A z, \quad x,z \in X, \alpha, \beta \in k \] 
\end{definition}
 
$\Lin(X,Y)$ ~-- множество линейных операторов из $X$ в $Y$. Также нам понадобится линейное пространство над $k$
\begin{gather*}
    \alpha \in k, A \in \Lin(X,Y), (\alpha A)(x) := \alpha Ax, \bZero(x) = 0 \text{ (0 в пространстве Y)} \\
    A, B \in \Lin(X,Y), (A+B)(x) := Ax + Bx 
\end{gather*}
Если $X = Y$, пишем только $\Lin(X)$.
\begin{example}[интегральный оператор]
    $C[a,b], K(s,t) \in C([a,b] \times [a,b])$
    \begin{gather*}
        f \in C[a,b], (\K f)(s) = \int^b_a k(s,t) f(t) dt \\
        (\K f)(s) \in C[a,b], \K \in \Lin(C[a,b])
    \end{gather*}
\end{example}
 
\begin{example}[оператор дифференцирования]
    $X = C^{(1)}[0,1] = \{ f: f^\prime \in C[0,1] \}$, $Y = C[0,1]$. $f \in X, D(f) = f^\prime,D \in \Lin(X,Y)$
\end{example}
 
\begin{example}[оператор вложения]
    $l^1 \subset l^2$, $x = \seq{x_n}^\infty_{n=1}, \sum^\infty_{n=1} |x_n| < + \infty, x \in l^1 \Rightarrow \sum^\infty_{n=1} |x_n|^2 < + \infty \Rightarrow x \in l^2$
    \begin{gather*} 
        Ax = x, A \text{ оператор вложения } l^1 \xhookrightarrow{A} l^2 \\
        \forall 1 \leq p_1 < p_2 \leq + \infty \Rightarrow l^{p_1} \xhookrightarrow{A} l^{p_2}, Ax = x \\
        A \in \Lin(l^{p_1}, l^{p_2})
    \end{gather*}
\end{example}
 
\begin{example}[оператор, но не линейный]x =
    $X$ ~-- линейное пространство, $x_0 \in X, x_0 \ne 0$, $Ax = x + x_0 \Rightarrow A$ ~-- не линейный.
\end{example}
 
Перед тем, как доказывать теорему, еще одно небольшое определение.
\begin{definition}[Выпуклое множество]
    $B \subset X, X$ ~-- линейное пространство. $B$ ~-- \textbf{ выпуклое }, если 
    \[ \forall x,z \in B, \forall t, 0 \leq t \leq 1 \Rightarrow tx + (1-t)z \in B \]
    то есть отрезок, соединяющий любые две точки, полностью лежит в этом множестве
\end{definition}
 
\begin{theorem}[простейшие свойства линейного оператора]
    $X,Y$ ~-- линейные пространства над $k$ ($\bR$ или $\bC$), $A \in \Lin(X,Y)$
    \begin{enumerate}
        \item $L \subset X, L$ ~-- подпространство в $X$ $\Rightarrow A(L)$ ~-- подпространство в $Y$ (образ подпространства ~-- подпространство)
        \item $M \subset Y, M$ ~-- подпространство в $Y \Rightarrow$ $\underbrace{A^{-1}(M)}_{\text{прообраз}}$ ~-- подпространство в $X$
        \item $B \subset X, B$ ~-- выпуклое $\Rightarrow A(B)$ ~-- выпуклое в $Y$
        \item $C \subset Y, C$ ~-- выпуклое $\Rightarrow A^{-1}(C)$ ~-- выпуклое в $X$
        \item пусть $A$ ~-- биекция $\Rightarrow A^{-1} \in \Lin(Y,X)$
    \end{enumerate}
\end{theorem}
Все 5 свойств доказывать не будем, покажем только несколько и скажем, что остальные доказываются аналогично.
\begin{proof}[1]
    $L$ ~-- подпространство, $y,v \in A(L), \alpha \in k$. Наша мечта ~-- проверить $(\stackrel{?}{\Rightarrow} \alpha y + v \in A(L))$, не обязательно писать $\alpha$ и $\beta$.
    \begin{gather*}
        \Rightarrow \, \exists x,y \in L : (Ax = y \land Au = v) \Rightarrow A(\alpha x + u) = \alpha A x + A u = \alpha y + v \\
        \alpha x + u \in L \Rightarrow A(\alpha x + u) \in A(L) \Rightarrow \alpha y + v \in A(L)
    \end{gather*}
\end{proof}
 
3 проверяется тютелька в тютельку как 1, а 2 ~-- как 4, поэтому проверим 4.
 
\begin{proof}[4]
    $C$ ~-- выпуклое, $x,u \in A^{-1}(C), 0 \leq t \leq 1$. 
    \begin{gather*}
        (y := Ax \land v := Au) \quad y,v \in C \Rightarrow ty + (1-t)v \in C \\
        A(tx + (1-t)u) = t Ax + (1-t) A u = ty + (1-t) v \in C \\
        \Rightarrow tx + (1-t)u \in A^{-1}(C) \Rightarrow A^{-1}(C) \text{ выпуклое }
    \end{gather*}
\end{proof}
 
\begin{proof}[5]
    $y, v \in Y \Rightarrow x = A^{-1}y, u = A^{-1}v \Rightarrow (Ax = y \land Au = v) \Rightarrow$
    \begin{gather*}
        \text{пусть } \alpha \in k, \quad  A(\alpha x + u) = \alpha A x + A u = \alpha y + v \Rightarrow \\
        \alpha x + u = A^{-1}(\alpha y + v) = \alpha A^{-1} y + A^{-1} v \Rightarrow \\
        A^{-1} \in \Lin(Y,X)
    \end{gather*}
\end{proof}
 
\begin{definition}[Ядро линейного оператора]
    $A \in \Lin(X,Y)$
    \[ \Ker A = \{ x \in X: Ax = 0 \} \text{ ~-- ядро } A \] 
    \[ \Imm A = \{ y \in Y: \, \exists x: Ax = y \} = A(X) \text{ ~-- образ } A \] 
\end{definition}
 
\begin{corollary}
    $X,Y$ ~-- линейные пространства, $\Rightarrow \Ker A$ ~-- подпространство в $X$, $\Imm A$ ~-- подпространство в $Y$.
\end{corollary}
 
\begin{definition}[произведение операторов]
    $X, Y,Z$ ~-- линейные пространства
    \[ X \stackrel{A}{\rightarrow} Y \stackrel{B}{\rightarrow} Z \] 
    $A \in \Lin(X,Y), B \in \Lin(Y,Z)$, $C = BA, C(x) := B(Ax), x \in X \Rightarrow C \in \Lin(X,Z), C$ ~-- произведение $BA$
\end{definition}
 
Всё самое тривиальное для операторов в линейных простаранствах мы вспомнили
 
\section{Линейные операторы в нормированных пространствах}
Линейные операторы в нормированных пространствах ~-- главный объект, который изучает функциональный анализ.
\begin{definition}[Огранисченный оператор]
    $(X, || \cdot||)$, $(Y, || \cdot ||), A \in \Lin(X,Y) $. $A$ ~-- \textbf{ ограниченный}, если $\forall C \subset X, C$ ~-- ограниченное $\Rightarrow A(C)$ ~-- ограниченное 
    в $Y$.
\end{definition}
Оказывается, для операторов ограниченность эквивалентна непрерывности. Казалось бы, ограниченность сильно слабее, но если к ней добавить  линейность, то будет аж непрерывность. \\
Обычно если в теореме 2 свойства, то говорят <<если и только если>>, а если условий несколько, то говорят <<равносильность>>. Подлые анголосаксы говорят Following Conditions are Equivalent.
 
\begin{theorem}[эквивалентность ограниченности и непрерывности линейного оператора]
    $(X, ||\cdot||), (Y, ||\cdot||), A \in \Lin(X,Y)$. Следующие условия равносильны (СУР) (FCE)
    \begin{enumerate}
        \item $A$ непрерывен в точке $0$ 
        \item $A$ непрерывен $\forall x \in X$  % (.) точка 
        \item $\exists C > 0 : ||Ax|| \leq C||x|| \, \forall x \in X $
        \item $A$ ограниченный
        \item $\exists r > 0 \: A(B_r(0))$ ~-- ограниченное множество в $Y$.
    \end{enumerate}
\end{theorem}
 
Доказательство очень простое, и, конечно, строится на линейности
\begin{proof}[$1 \Rightarrow 2$]
    $A$ непрерывен в точке $0$.
    Пусть $\varepsilon > 0 \, \exists \delta > 0, \, ||x|| < \delta \Rightarrow ||Ax|| < \varepsilon \: (A(\bZero)) = \bZero$.
    утверждается, что те же самые $\varepsilon$ и $\delta$ подходят.
    \begin{gather*}
        \text{пусть } x_0 \in X, \text{проверим, что } A \text{ непрерывен в } x_0 \\
        \text{пусть } ||x-x_0|| < \delta \Rightarrow ||A(x-x_0)|| < \varepsilon \Rightarrow ||Ax - Ax_0|| < \varepsilon
    \end{gather*}
\end{proof}
$2 \Rightarrow 1$ очевидно 
\begin{proof}[$1 \Rightarrow 3$]
    Пусть $\varepsilon > 0 \, \exists \delta > 0 : ||x|| < \delta \Rightarrow ||Ax|| < \varepsilon$.
    \begin{gather*}
        z \in X, z \ne 0 \quad x = \frac{z}{||z||} \cdot \delta \Rightarrow ||x|| = \delta \Rightarrow ||Ax|| < \varepsilon \\
        \Rightarrow ||A\left( \frac{z}{||z||} \cdot \delta\right)|| < \varepsilon \Rightarrow ||Az|| < \frac{\varepsilon}{\delta} ||z|| \text{т.е.} C = \frac{\varepsilon}{\delta}
    \end{gather*}
\end{proof}
 
\begin{proof}[$3 \Rightarrow 4$]
    $B \subset X$, $B$ ~-- ограниченное, то есть $\exists M > 0 : (\forall x \in B \land ||x|| < M) \stackrel{3}{\Rightarrow} ||Ax|| \leq C||x|| \leq CM \, \forall x \in B \Rightarrow 
    \{ A(B) \}$ ~-- ограниченное.
\end{proof}
$4 \Rightarrow 5$ очевидно ($B_r(0)$ ~-- ограниченное)
\begin{proof}[$5 \Rightarrow 1$]
    $\exists R > 0 \, A(B^x_r(0)) \subset B^y_R(0)$
    \begin{gather*}
        ||x|| < r \Rightarrow ||Ax|| < R \\
        \intertext{непрерывность в 0 означает} 
        \text{ пусть } \varepsilon > 0 \quad ||x|| < \delta(\varepsilon) \Rightarrow ||Ax|| < \varepsilon \\
        \delta(\varepsilon) = \varepsilon \cdot \frac{r}{R}\\
        ||z|| < \varepsilon \cdot \frac{r}{R} \Rightarrow ||z \cdot \frac{R}{\varepsilon} || < r \Rightarrow ||A \left( z \cdot \frac{R}{z} \right) || < R \Rightarrow ||Az|| < \varepsilon
    \end{gather*}
\end{proof}
 
с помощью теоремы, которую мы только что доказали, введём норму в этом пространстве.
\begin{definition}[норма оператора]
    $(X, ||\cdot||), (Y, ||\cdot||)$
    \[ \underbrace{\B(X,Y)}_{\text{bounded}} = \{ A \in \Lin(X,Y) \land A \text{ ~-- ограниченный} \} \]
    $A \in \B(X,Y)$
    \[ ||A|| = \inf \{ C : C > 0 \land \norm{Ax} \leq C\norm{x} \: \forall x \in X \} \]
    то бишь точная нижняя грань множества величин, на которые наш оператор увеличивает норму элемента.
\end{definition}
 
Раз мы так объявили норму, то надо проверять аксиомы нормы. 
 
\begin{statement}
    $(X, || \cdot||), (Y, ||\cdot||), A \in \B(X,Y)$
    \begin{enumerate}
        \item $\forall x \in X \, ||Ax|| \leq ||A||  ||x||$ (то есть $\inf$ в определении нормы $=\min)$
        \item $||A||$ удовлетворяет аксиомам нормы
    \end{enumerate}
\end{statement}
 
\begin{proof}
    $x$ - фиксирован, $\Rightarrow \, \forall c > ||A||, ||Ax|| \leq C ||x|| \Rightarrow ||Ax|| \leq ||A|| \cdot ||x||$. Был фиксирован, теперь любой,
     первое утверждение доказано. Теперь второе. 
     \begin{gather*}
        \alpha \in k, \alpha \ne 0, x \in X, x \text{ ~-- фиксирован } \\
        (\alpha A) (x) = \alpha A x \\
        \forall x \in X \quad ||(\alpha A)(x) || = ||\alpha \cdot Ax|| = |\alpha| \cdot ||A x|| \leq |\alpha| \cdot ||A|| \cdot ||x|| \\
        \Rightarrow ||\alpha A|| \leq |\alpha| ||A||
     \end{gather*}
     Очевидное замечание по слёзной просьбе двух студенток, которые ничего не понимали.
     Если мы докажем $||Ax|| \leq M||x|| \, \forall x \in X$, то $||A|| \leq M$
 
     \begin{gather*}
        \Rightarrow \norm{\frac{1}{\alpha}(\alpha A)} \leq \frac{1}{|\alpha|} ||\alpha A|| \Rightarrow |\alpha| || A|| \leq ||\alpha A|| \\
        \Rightarrow ||\alpha A || = |\alpha| ||A|| \\
        A,B \in \B(X,Y), x \in X 
     \end{gather*}
     \begin{multline*}
        ||(A+B)(x)|| = ||Ax + Bx|| \leq ||Ax|| + ||Bx|| \leq ||A|| \cdot ||x|| + ||B|| \cdot ||x|| = \\
        = (||A|| + ||B||) ||x|| \quad \forall x \in X \\
        \Rightarrow ||A+B|| \leq ||A||+ ||B||
     \end{multline*}
     Как только есть какая-то константа, то настоящая норма меньше или равна этой константы. $||A|| = 0 \Rightarrow \, \forall x \in X \, ||Ax|| \leq ||A|| \cdot ||x|| = 0$.
     $\Rightarrow Ax = 0 \, \forall x \in X \Rightarrow A = \bZero \Rightarrow ||A||$ ~-- настоящая норма
\end{proof}
 
\begin{theorem}[вычисление нормы непрерывного оператора]
    $(X, ||\cdot||), (Y, ||\cdot||), A \in \B(X,Y) \Rightarrow$ 
    \[ ||A|| = \underbrace{\sup_{\{||x|| \leq 1\}} ||Ax||}_a = \underbrace{\sup_{\{||x|| < 1\}} ||Ax||}_b = \underbrace{\sup_{\{||x|| = 1\}} ||Ax||}_c = \underbrace{\sup_{\{x \in X, x \ne 0\}} \frac{||Ax||}{||x||}}_d \]
\end{theorem}
 
\begin{proof}
    Очевидно $a \geq b, a \geq c, d \geq c$.
    Докажем $||A|| \geq a \geq b \geq ||A||, \quad ||A|| \geq d \geq c \geq ||A||$.
       \[ ||Ax|| \leq ||A|| \cdot ||x|| \leq ||A|| \quad \forall x, ||x|| \geq 1 \Rightarrow \sup_{\{||x|| \geq 1 \}} ||Ax|| \leq ||A|| \Rightarrow a \leq ||A|| \]
    Доказали $||A|| \geq a$. \\
    Пусть $\varepsilon > 0 \, z \in X, z \ne 0 \Rightarrow \left| \left| \frac{z}{||z||(1+\varepsilon)} \right| \right| = \frac{1}{1+\varepsilon} < 1$
    \begin{gather*}
        \left| \left| A(\frac{z}{||z||(1+\varepsilon)}) \right| \right| \leq b \Rightarrow ||Az|| \leq b(1+\varepsilon) ||z|| \quad \forall z \in X \\
        \Rightarrow ||A|| \leq b(1 + \varepsilon) \, \forall \varepsilon > 0 \Rightarrow ||A|| \leq b
    \end{gather*}
    Закончили с первой цепочкой неравенств. \\
    Пусть $x \ne 0 \Rightarrow ||Ax|| \leq ||A|| \cdot ||x|| \Rightarrow \frac{||Ax||}{||x||} \leq ||A|| \Rightarrow d = \sup_{\{x \ne 0 \}} \frac{||Ax||}{||x||} \leq ||A||$.
    \begin{gather*}
        \text{пусть } z \in X, z \ne 0, \, \left| \left| \frac{z}{||z||} \right|\right| = 1 \Rightarrow ||A\left(\frac{z}{||z||}\right)|| \leq c \Rightarrow ||Az|| \leq C||z|| \forall z \in X
        \intertext{c ~-- супремум по единичной сфере}
        \Rightarrow ||A|| \leq C
    \end{gather*}
\end{proof}
 
\begin{example}
    $C[a,b]$, $h(x) \in C[a,b]$ ~-- фиксированная функция. $f \in  C[a,b], M_h(f) := h(x) \cdot f(x)$.
    \[ M_h \in \Lin(C[a,b]) \]
 
\end{example}
Проверим, что он непрерывен и сосчитаем его норму.
\begin{proof}
    \begin{multline*}
        ||M_h(f)||_\infty = \max_{x \in [a,b]} | h(x) \cdot f(x)| \leq \max_{x \in [a,b]} |h(x)| \cdot \max_{x \in [a,b]} |f(x)| = ||h||_\infty \cdot ||f||_\infty \\
        \Rightarrow M_h \in \B(C[a,b]), ||M_h||_{\B(C[a,b])} \leq ||h||_\infty
    \end{multline*}
    получили непрерывность; раз есть общая константа, не зависящая от $f$, то мы получаем и оценку для нормы
    \begin{gather*}
        \chi_{[a,b]}(x) = 1 \, \forall x \in [a,b], \chi_{[a,b]} \in C[a,b], ||\chi_{[a,b]} ||_\infty = 1 \\
        ||M_h|| \geq ||M_h(f)|| \forall f, ||f|| = 1 \Rightarrow ||M_h|| \geq || M_h(\chi_{[a,b]}) ||_\infty = ||h||_\infty \\
        \Rightarrow ||M_h||_{\B(C[a,b])} = ||h||_\infty
    \end{gather*}
\end{proof}
 
Теперь посмотрим на оператор дифференцирования, это очень важный пример.
\begin{example}
    $Y = C[a,b], X = \{f: \, \exists f^\prime \in C[a,b] \}$, $0 \leq a \leq b$
    \begin{gather*}
        X \subset Y, X \text{ ~-- подпространство } Y, \text{ то есть } \\
        ||f||_X = ||f||_Y = \max_{x \in [a,b]} |f(x)| \\
        D(f) = f^\prime \Rightarrow D \in \Lin(X,Y), \\
        D(x^n) = n x^{n-1} \quad \sup_{n \in \bN} \frac{||D(x^n)||}{||x^n||} = \sup_{n \in \bN} \frac{nb^{n-1}}{b^n} = +\infty
    \end{gather*}
    при таком определении нормы оператор дифференцирования  $D$ не непрерывен.
\end{example}
 
\begin{example}
    $Y = C[a,b], X = C^{(1)}[a,b]$
    \begin{gather*}
        ||f||_X = \max \{ ||f||_\infty, ||f^\prime||_\infty \} \\
        D(f) = f^\prime \quad ||D(f)|| = ||f^\prime||_\infty = \max_{x \in [a,b]} |f^\prime(x)| \leq \underbrace{\max \{ ||f||_\infty, ||f||_\infty \}}_{||f||_X} \\ 
        \Rightarrow D \in \B(X, Y), ||D|| \leq 1
    \end{gather*}
\end{example}
 
 
\begin{theorem}[вложение пространств в $l^p$]
    Пусть $1 \leq p_1 < p_2 \leq +\infty$. $x \in l^p$. Рассмотрим оператор вложения $Ax = x \Rightarrow A \in \B(l^{p_1}, l^{p_2}), ||A||=1$.
\end{theorem}
 
\begin{proof}
    То, что он линейный, мы уже обсуждали, это очевидно. Удобно будет рассматривать последовательности из единичной сферы.
    $x \in l^p, x = \seq{x_n}^\infty_{n=1}, x_n \in \bC$. $||x||_p = \left(\sum^\infty_{n=1} |x_n|^p \right)^{\frac{1}{p}}, 1 \leq p < + \infty$.  Возьмём не просто последовательность из $l^{p_1}$, но и такую, что
    $\norm{x}_{p_1} = 1 \Rightarrow \sum^\infty_{n=1} |x_n|^{p_1} = 1 \quad Ax = x$.
    \begin{gather*}
        \Rightarrow |x_n| \leq 1 \Rightarrow (|x_n|^{p_2}) < |x_n|^{p_1} \\
        \norm{Ax}_{p_2} = \left( \sum^\infty_{n=1} |x_n|^{p_2} \right)^{\frac{1}{p_2}} \leq \left( \sum^\infty_{n=1} |x_n|^{p_1} \right)^{\frac{1}{p_2}} = 1 \Rightarrow A \in \B(l^{p_1}, l^{p_2}) \\
        \norm{A} = \sup_{ \{ \norm{x}_{p_1} = 1 \}} \norm{Ax}_{p_2} \leq 1 \Rightarrow \norm{A}_{\B(l^{p_1}, l^{p_2})} \leq 1 \quad \text{ при } p_2 < + \infty \\
        \text{ теперь } p_2 = +\infty \norm{x}_{p_1} = 1 \Rightarrow \sup_{n \in \bN} |x_n| \leq \norm{x}_{p_1} \Rightarrow \norm{x}_\infty \leq \norm{x}_{p_1} \Rightarrow  \\
        A \in \B(l^{p_1}, l^{p_2}) \norm{A} \leq 1 \\
    \end{gather*}
    если $e_1 = (1, 0, \ldots)$, $\norm{e_1}_p = 1 \: \forall p : 1 \leq p \leq + \infty$
    \[ \norm{A} = \sup_{\{\norm{x}_{p_1} = 1\}} \norm{Ax}_{p_2} \geq \norm{Ae_1}_{p_2} = 1 \Rightarrow \norm{A}_{\B(l^{p_1}, l^{p_2})} = 1 \quad \forall p_1 < p_2 \]
\end{proof}
 
Посмотрим теперь на похожую теорему для больших пространств $L^p$.
 
\begin{theorem}[вложение пространств в $L^p(\mu)$ для конечной меры]
    $(X,U, \mu), 1 \leq p_1 < p_2 \leq +\infty, \mu(X) < +\infty$. Рассмотрим $f \in L^{p_2}, Af = f \Rightarrow A \in \B(L^{p_2}, L^{p_1})$.
    $\norm{A} = (\mu(X))^{\frac{1}{p_1} - \frac{1}{p_2}}, \left( \frac{1}{\infty} = 0 \right)$
\end{theorem}
 
\begin{proof}
    Начнём с самого простого случая. То есть что называлось существенно ограниченными функциями.
    $p_2 = \infty, f \in L^\infty(\mu), |f(x)| \leq \norm{f}_\infty$ п.в. для $x \in X$ по $\mu$.
    \begin{gather*}
        \norm{Af}_{p_1} = \norm{f}_{p_1} = \left( \int_X |f|^{p_1} d\mu \right)^{\frac{1}{p_1}} \leq \norm{f}_\infty \left( \int_X d\mu \right)^{\frac{1}{p_1}} = \norm{f}_\infty \mu(X)^{\frac{1}{p_1}}
        \intertext{ Вот у нас получилась константа, которая обслуживает все функции $f$. Тогда, во-первых, оператор непрерывен, а во-вторых, это и есть оценка для нормы}
        \Rightarrow A \in \B(L^\infty, L^{p_1}), \norm{A} \leq (\mu(X))^{\frac{1}{p_1}} \\
        \text{ пусть } p_2 < + \infty, f \in L^{p_2}, \left( \int_X |f|^{p_2} d\mu \right)^{\frac{1}{p_2}} = \norm{f}_{p_2} \\
        \norm{Af}_{p_1} = \norm{f}_{p_1} = \left( \int_X |f|^{p_1} d\mu \right)^{\frac{1}{p_1}} \stackrel{\text{н. Гёльдера}}{\leq} \left[ \left( \int_X |f|^{p_2} d\mu\right)^{\frac{1}{p_2}} \left( \int_X \mathbb{1}^q d\mu \right)^{\frac{1}{q}} \right]^{\frac{1}{p_1}} = \\ %цвет нужен
        p = \frac{p_2}{p_1}, \frac{1}{q} = 1 - \frac{1}{p} = 1 - \frac{p_1}{p_2} \\
        = \left( \int_X |f|^{p_2} d\mu \right)^{\frac{1}{p_2}} \cdot (\mu(X))^{\left(1 - \frac{p_1}{p_2}\right) \frac{1}{p_1}} = \norm{f}_{p_2} (\mu(X))^{\frac{1}{p_1} - \frac{1}{p_2}} \\
        \Rightarrow A \in \B(L^{p_2}, L^{p_1}), \norm{A} \leq (\mu(X))^{\frac{1}{p_1} - \frac{1}{p_2}}
    \end{gather*}
    Почти всё готово. Мы оценили норму сверху, и утверждается, что на самом деле имеет место равенство. На какой пробной функции получить неравенство с другой стороны? Наверное, все уже догадались.
    Раз есть $\sup$, то мы можем подставить какую-то конкретную функцию.
    $p_2 < +\infty, \chi_{X}(x) \equiv 1$
    \begin{multline*}
        \norm{A} = \sup_{f \ne \mathbb{0}} \frac{\norm{Af}_{p_1}}{\norm{f}_{p_2}} \geq \frac{\norm{A(\chi_X)}_{p_1}}{\norm{\chi_X}_{p_2}} = \frac{\left( \int_X \chi_X^{p_1} d\mu \right)^{\frac{1}{p_1}}}{\left( \int_X \chi_X^{p_2} d\mu \right)^{\frac{1}{p_2}}} = \\
        = \frac{(\mu(X))^{\frac{1}{p_1}}}{\mu(X)^{\frac{1}{p_2}}} = \mu(X)^{\frac{1}{p_1}-\frac{1}{p_2}}
    \end{multline*}
    если $p_2 = \infty, \norm{\chi_x}_\infty = 1 \Rightarrow \norm{A}_{\B(L^\infty,L^{p_1})} \geq \mu(X)^{\frac{1}{p_1}}$ 
\end{proof}
 
Позже вычислим норму интегрального оператора, который часто встречается в анализе и в матфизике.
 
\begin{theorem}[полнота пространства операторов, действующих в банахово пространство]
    $(X, \norm{\cdot})$ -- нормированное, $(Y, \norm{\cdot})$ -- банахово $\Rightarrow \B(X,Y)$ -- банахово.
\end{theorem}
 
\begin{proof}
    Тут без хитростей. По определению возьмём фундаментальную последовательность и покажем, что у нее есть предел. Сначала надо добыть оператор, который будет претендентом на звание предела.
    $\seq{A_n}_{n=1}^\infty$ -- фундаментальная, $A_n \in \B(X,Y)$. Пусть $\varepsilon > 0 \: \exists N \in \bN \: (n > N \land m > N) \Rightarrow \norm{A_n -A_m} < \varepsilon$.
    $x \in X, x$ -- фиксирован, $\Rightarrow \norm{A_n x - A_m x} = \norm{(A_n-A_m)x} < \varepsilon \norm{x}$. Тогда $\seq{A_n x}^\infty_{n=1}$ -- фундаментальная в $Y, Y$ -- банахово $\Rightarrow$
    \begin{gather*}
        \exists \liml_{n \to \infty} A_n x \in Y, Ax := \liml_{n \to \infty} A_n x \text{ поточечный предел} \\
        \lim \text{ -- линейная } \Rightarrow A \in \Lin(X,Y) \\
        x \text{ -- фиксирован } \norm{A_n x - A_m x} < \varepsilon \norm{x}, \text{ пусть } m \to \infty \\
        \Rightarrow \norm{A_n x - Ax} \leq \varepsilon \norm{x} \quad \forall x \in X \\
        \Rightarrow A_n - A \in \B(X,Y), \norm{A_n - A} \leq \varepsilon \Rightarrow A = (A - A_n) +  A_n \Rightarrow A \in \B(X,Y)
    \end{gather*}
\end{proof}
 
Поговорим немного о линейных функционалах. Вы только не думайте, что мы покидаем линейые операторы, это всё-таки главный объект изучения функционального анализа.
 
\section{Линейные функционалы} 
 
\begin{definition}[линейный функционал]
    $X$ -- линейное пространство над $k$ ($\bR$ или $\bC$). $\Lin(X,k)$ -- линейные функционалы на $X$
\end{definition}
 
\begin{definition}[сопряжённое пространство]
    $(X, \norm{\cdot}), X^* = \B(X, \bC)$  (или же $X^* = \B(X, \bR))$ -- сопряжённое пространство.
    $X^*$ -- линейные \textbf{НЕПРЕРЫВНЫЕ} функционалы.
\end{definition}
 
Про неперывность надо помнить. На экзамене часто спрашивают, что такое сопряжённое пространство, и не могут выпытать непрерывность. Что делают с такими студентами? Выгоняют.
 
\begin{corollary}
    $(X, \norm{\cdot}), f \in X^* \Rightarrow$
    \[ \norm{f} = \sup_{\seq{\norm{x} \leq 1}} |f(x)| = \sup_{\seq{\norm{x} < 1}} |f(x)| = \sup_{\seq{\norm{x} = 1}} |f(x)| = \sup_{x \in X, x \ne 0} \frac{|f(x)|}{\norm{x}} \]
\end{corollary}
 
\begin{corollary}
    $(X, \norm{\cdot})$ $\Rightarrow X^*$ -- банахово
\end{corollary}
\begin{proof}
    $\bR$ и $\bC$ -- полные $\Rightarrow \B(X, \bC)$ -- банахово ($\Rightarrow \B(X, \bR)$ -- банахово).
\end{proof}
 
\begin{example}
    $X = l^p, (1 \leq p \leq + \infty), i \in \bN$ -- фиксированное число 
    \begin{gather*}
        x \in l^p \Rightarrow x = \seq{x_n}^\infty_{n=1}, x_n \in \bC, f(x) := x_i \Rightarrow f \in X^*, \norm{f} = 1 \\
        |f(x)|  = |x_i| \leq \left(\sum^\infty_{n=1} |x_n|^p \right)^{\frac{1}{p}}  \text{ при } 1 \leq p < +\infty \text{ и } \\
         \leq \sup_n \norm{x_n} = \norm{x}_\infty \text{ при } p = +\infty\\
        \Rightarrow f \in \B(X, \bC) = X^*, ||f|| \leq 1 \\
        \norm{f} = \sup_{\seq{\norm{x} = 1}} |f(x)| \geq |f(e_i)| = 1
    \end{gather*}
\end{example}
Со временем мы сосчитаем, что такое сопряженное пространство к $l^p$ для конечных $p$. По секрету, это $l^q$, где $p$ и $q$ -- сопряжены.
 
Почему всегда рассматривается компакт? Потому что на компакте функция достигает свой максимум, и иначе непонятно, как норму вводить.
\begin{example}
    $C(K) = \{ f: K \rightarrow \bC \land f \text{ непрерывные } \}, x_0 \in K, K$ -- компакт. \\
    $f \in C(K), G(f) := f(x_0) \Rightarrow G \in X^*, \norm{G} = 1$ (функционал значения в точке,подлые англосаксы говорят point evaluation).
    \begin{gather*}
        G \in \Lin(C(K), \bC) \\
        f \in C(K), |G(f)| = |f(x_0)| \leq \sup_{x \in K} |f(x)| = \norm{f}_{C(k)} \Rightarrow \\
        G \in X^*, \norm{G} \leq 1 \\
        \begin{cases}
            \chi_K(x) = 1, \chi_K \in C(K), ||\chi_K|| = 1, \chi_K(x_0) = 1 \\
            \Rightarrow \norm{G} =\sup_{\seq{\norm{f} = 1}} |G(f)| \geq |G(\chi_K)| = 1
        \end{cases} \Rightarrow \norm{G} = 1
    \end{gather*} 
\end{example}
 
Когда-то мы опишем пространство непрерывных функций, но доказывать, почему оно так выглядит, не будем, ибо это очень сложно, и придётся просто поверить в это описание.
Сейчас докажем теорему про норму интегрального оператора в $C[a,b]$. Мы ей даже когда-то нескоро воспользуемся.
 
\begin{theorem}
    $C[a,b] = \seq{f | f: [a,b] \rightarrow \bR, f \text{ непрерывная }}$. Ядро интегрального оператора $k(s,t) \in C([a,b] \times [a,b])$, пусть $f \in C[a,b]$.
    \[ (\K f)(s) := \int^b_a k(s,t)f(t) dt \quad \text{ при } s \in [a,b] \Rightarrow \] 
    $\K \in \B(C[a,b]), \norm{\K} = \max_{a \leq s \leq b} \int^b_a |k(s,t)| dt$
\end{theorem}
 
Доказательство начнём с важной леммы, помогающий вычислить норму линейного функционала. Когда мы сосчитаем норму линейного функционала, то будет очень нетрудно применить
 это для вычисления нормы линейного оператора.
\begin{lemma}
    $\varphi(t) \in C[a,b], \varphi$ -- фиксирована. $f \in C[a,b], G(f) := \int^b_a f(t) \varphi(t)dt \Rightarrow G \in (C[a,b])^*, \norm{G} = \int^b_a | \varphi(t)| dt$. 
\end{lemma}
 
\begin{proof}[Доказательство леммы]
    Оценка сверху совершенно тривиальна.
    $f \in C[a,b]$
    \begin{multline*}
        |G(f)| = \left| \int^b_a f(t) \varphi(t) dt \right| \leq \int^b_a |f(t)| |\varphi(t)| dt \leq \max_{t \in [a,b]} |f(t)| \cdot \int^b_a |\varphi(t)| dt = \\ =
        \norm{f}_\infty \int^b_a |\varphi(t)| dt \Rightarrow \\
        G \in (C[a,b])^*, \norm{G} \leq \int^b_a |\varphi(t)| dt
    \end{multline*}
    Теперь оценка $\norm{G}$ снизу. Сначала тривиальные замечания. Если $\varphi(t) \geq 0 \: \forall t \in [a,b]$, то $\chi_{[a,b]}(x) \equiv 1$
    \[ |G(\chi_[a,b])| = \left| \int^b_a \varphi(t) dt \right| = \int^b_a \varphi(t) dt \]
    Если $\varphi(t) \leq 0 \, \forall t \in [a,b]$ -- то же самое.
    \[ g(t) = \sign \, \varphi(t) = \begin{cases}
        1 & \varphi(t) > 0 \\
        -1 & \varphi(t) < 0 \\
        0 &\varphi(t) = 0
    \end{cases} \]
    $G(g) = \int^b_a |\varphi(t)| dt$, но $g \notin C[a,b]$.
    До сих пор мы всегда находили пробную функцию, на котором достигался $\sup$, а здесь такого элемента нет.
    Поэтому будем приближать $\varphi$ непрерывными функциями  с точностью до $\varepsilon$, вот такая идея. \\
    Пусть $\varepsilon > 0, \varphi \in C[a,b] \Rightarrow \varphi$ -- равномерно непрерывна на $[a,b] \Rightarrow$
    \[ \exists \delta > 0 \: |s-t| < \delta \Rightarrow |\varphi(t) - \varphi(s)| < \varepsilon \quad a \leq s, t \leq b \]
    $a = t_0 < t_1 < \ldots < t_n = b, t_k - t_{k-1} < \delta$. \\
    Рассмотрим $\seq{\Delta_j}^n_{j=1}$. $\Delta_j$ -- интервалы $[t_{k-1}, t_k]$.
    Нумерация будет не по порядку, как сперва может показаться, а совершенно другая, и она никак не будет зависеть от расположения на отрезке.
     Разобьём интервал на 2 сорта. Первый -- где функция положительна или отрицательна, то есть не меняет знак. Второй -- где меняет знак или обращается в 0.
    $\Delta_1, \ldots, \Delta_r$ -- те интервалы, на которых $\varphi(t) > 0, t \in \Delta_j$ или $\varphi(t) < 0, t \in \Delta_j$ $(1 \leq j \leq r)$ \\
    $\Delta_{r+1}, \ldots, \Delta_n$ -- те интервалы, для которых $\exists s \in \Delta_j : \varphi(s) = 0, n \geq j > r$. \\
    \begin{gather*}
        \text{пусть } t \in \Delta_j, j > r \Rightarrow \, \exists s \in \Delta_j, \varphi(s) = 0 \Rightarrow \\
        |\varphi(t)| = |\varphi(t) - \varphi(s)| < \varepsilon \Rightarrow \int_{\Delta_j} |\varphi(t)| dt < \varepsilon | \Delta_j| \\
        \Rightarrow \int_{\bigcup^n_{j=r+1} \Delta_j} \abs{\varphi(t)}dt\leq \varepsilon \left( \sum^n_{j=r+1} |\Delta_j| \right) \leq \varepsilon(b-a)\\
        h(t) = \begin{cases}
            \sign \, \varphi(t), t \in \Delta_j \quad 1 \leq j \leq r \\
            \text{линейная на } \Delta_j \quad j > r \\
            \text{если } [a,t_1] \in \Delta_j, j >r, \text{ то } h(a) = 0 \\
            \text{если } [t_{n-1}, b] \in \Delta_j, j >r, \text{ то } h(b) = 0
        \end{cases}
        h \in C[a,b], |h(t)| \leq 1 \\
    \end{gather*}
    \begin{multline*}
        \norm{G} = \sup_{\seq{\norm{f} \leq 1}} |G(f)| \geq |G(h)| = \left| \int^b_a h(t) \varphi(t) dt \right| = \\
        = \left| \int_{\bigcup^r_{j=1} \Delta_j} h(t) \varphi(t) dt + \int_{\bigcup^n_{j=r+1} \Delta_j} h(t) \varphi(t) dt \right| = \\
        = \left| \int_{\bigcup^r_{j=1} \Delta_j} |\varphi(t)| dt + \int_{\bigcup^n_{j=r+1} \Delta_j} h(t) \varphi(t) dt \right| \geq \\
        \geq \int_{\bigcup^r_{j=1} \Delta_j} |\varphi(t)| dt - \int_{\bigcup^n_{j=r+1} \Delta_j} |h(t)| |\varphi(t)| dt \geq \\
        \geq \int_{\bigcup^r_{j=1} \Delta_j} |\varphi(t)| dt - \int_{\bigcup^n_{j=r+1} \Delta_j} |\varphi(t)| dt = \int^b_a |\varphi(t)| dt - 2 \int_{\bigcup^n_{j=r+1} \Delta_j} |\varphi(t)|dt \geq \\
        \geq \int^b_a |\varphi(t)| dt - 2 \varepsilon(b-a) \quad \forall \varepsilon > 0
    \end{multline*}
    $\Rightarrow ||G|| \geq \int^b_a |\varphi(t)| dt $
\end{proof}
 
Главной частью доказательства теоремы было доказательство теоремы. Вернёмся к теореме.
\begin{proof}
    Оценим сначала норму оператора сверху. $( \K f)(s) = \int^b_a k(s,t) f(t) dt, f \in C[a,b]$. $M = \max_{a \leq s \leq b} \int^b_a |k(s,t)| dt$. Мы как раз
    хотим показать, что норма оператора будет равна $M$.
       \[ |(Kf)(s)| \leq \int_a^b |k(s,t)| |f(t)| dt \leq ||f||_\infty \int^b_a |k(s,t) |dt \leq M \norm{f}_\infty \]
       \[ \norm{\K f} = \max_{s} | \K f(s)| \leq < M \cdot \norm{f} \: \forall f \in C[a,b] \Rightarrow \K \in \B(C[a,b]) \]
       $\norm{K}_{\B(C[a,b])} \leq M$ \\
       Теперь оценим $\norm{\K}$ снизу. 
       \begin{gather*}
            g(s) = \int^b_a |k(s,t)| dt \Rightarrow g \in C[a,b] \Rightarrow \\
            \exists s_0 \: g(s_0) = \max g(s) \Rightarrow g(s_0) = M \\
            \intertext{применим к произвольной непрерывной функции оператор} 
            f \in C[a,b], \norm{(\K f)(s)}_\infty = \max_{a \leq s \leq b} |\K f(s)| \geq |(\K f)(s_0)| = \left| \int^b_a k(s_0,t) f(t) dt \right| = |G(f)|
       \end{gather*}
       где $\varphi(t) = K(s_0,t), G(f) = \int^b_a k(s_0,t) f(t) dt$.
       \begin{multline*}
            \norm{\K} = \sup_{\seq{\norm{f} \leq 1}} \norm{\K(f)} \geq \sup_{\seq{\norm{f} \leq 1}} |G(f)| = \norm{G}_{(C[a,b])^*} \stackrel{\text{лемма}}{=} \\
            \int^b_a |\varphi(t)| dt = M \Rightarrow \norm{K} = M
       \end{multline*}
\end{proof}
 
От сопряжённых пространств мы не уходим, а наоборот, углубляемся в них.
 
\section{Изоморфные линейные пространства}
 
\begin{definition}[изоморфность пространств]
    $(X, \norm{\cdot})$, $(Y, \norm{\cdot})$ -- \textbf{ линейно изоморфны}, если $\exists A \in \B(X,Y), \: \exists A^{-1} \in \B(Y,X)$. $A$ -- \textbf{линейный изоморфизм}
\end{definition}
 
\begin{remark}
    <<Изоморфность>> -- отношение эквивалентности на множестве нормированных пространств.
\end{remark}
 
Когда можно сказать, что два пространства изоморфны? 
\begin{theorem}[критерий линейного изоморфизма]
    $(X, \norm{\cdot}), (Y, \norm{\cdot}), A \in \Lin(X,Y), A(X) = Y$ (то есть $A$ -- сюръекция).\\
    $A$ -- линейный изоморфизм $\Leftrightarrow \text{ пусть } 0 < c_1 < C_2 < + \infty$ т.ч. $c_1\norm{x} \leq \norm{Ax} \leq C_2 \norm{x}, \: \forall x \in X$
\end{theorem}
 
\begin{proof}
    $\Rightarrow$   
    \begin{gather*}
        A \in \B(X,Y) \Rightarrow \norm{Ax} \leq \norm{A} \cdot\norm{X} \: \forall x \in X, C_2 = \norm{A} \\
        \exists A^{-1} \B(Y,X) \Rightarrow \norm{A^{-1}y} \leq \norm{A^{-1}} \norm{y} \: \forall y \in Y \\
        \text{пусть } x \in X, y = Ax \Rightarrow
        \norm{A^{-1}(Ax)} \leq \norm{A^{-1}} \cdot \norm{Ax} \Rightarrow \\
        \frac{1}{\norm{A^{-1}}} \cdot \norm{x} \leq \norm{Ax} \quad c_1 = \frac{1}{\norm{A^{-1}}}
    \end{gather*}
    $\Leftarrow$ \\
    $\norm{Ax} \leq C_2 \norm{x} \Rightarrow A \in \B(X,Y) (\norm{A} \leq C_2)$. Теперь проверим, что $A$ -- инъекция.
    Без неравенства снизу мы сейчас как раз выведем, что образы различных иксов различны. 
    Пусть $Ax_1 = Ax_2 \Rightarrow A(x_1 - x_2) = 0$
    \begin{gather*}
        0 = \norm{A(x_1-x_2)} \geq c \norm{x_1 - x_2} \Rightarrow x_1 = x_2 \Rightarrow A \text{ -- биекция } \\
        \stackrel{\text{доказали}}{\Rightarrow} \: \exists A^{-1} \in \Lin(Y,X) \\
        \begin{cases}
            c_1 \norm{x} \leq \norm{Ax} \: \forall x \in X \\
            \text{пусть } y \in Y, x = A^{-1}y
        \end{cases} \Rightarrow \\
         c_1\norm{A^{-1}y} \leq \norm{y} \Rightarrow \norm{A^{-1}y} \leq \frac{1}{c_1}\norm{y} \Rightarrow A^{-1} \in \B(Y,X) \left(\norm{A^{-1}} \leq \frac{1}{c_1}\right)
    \end{gather*}
\end{proof}
Раз нам предстоит потом долгий разговор про обратные операторы, сразу отметим некоторое следствия из доказательства теоремы, чтобы не возвращаться к нему потом.
\begin{corollary}[из доказательства теоремы]
    $(X, \norm{\cdot}), (Y, \norm{\cdot}), A \in \Lin(X,Y), A(X) = Y$
    \[\exists A^{-1} \in \B(Y,X) \Rightarrow \exists c > 0 : \norm{Ax} \geq c \norm{x} \: \forall x \in X \]
\end{corollary}
\begin{proof}
    Следует из доказательства теоремы.
\end{proof}
 
 
Часто бывает, что на одном и том же пространстве определены две различные нормы. Какие же нормы будут называться эквивалентными?
 
\begin{definition}
    $X$ -- линейное пространство, $\norm{\cdot}_1, \norm{\cdot}_2$ -- две нормы на $X$. $\norm{\cdot}_1$ эквивалентна $\norm{\cdot}_2$, если 
    \[ \liml_{n \to \infty} \norm{x_n - x_0}_1 = 0 \Leftrightarrow \liml_{n \to \infty} \norm{x_n - x_0}_2 = 0 \]
    По-другому можно сказать, что топологии, которые задают эти  нормы, одинаковые: $\Leftrightarrow G \subset X, G$ -- открытое в $(X, \norm{\cdot}_1) \Leftrightarrow G$ -- открытое в $(X, \norm{\cdot}_2)$
\end{definition}
 
\begin{corollary}
    $X$ -- линейное, $\norm{\cdot}_1, \norm{\cdot}_2$ -- нормы на $X$. $\norm{\cdot}_1$ эквивалентна $\norm{\cdot} \Leftrightarrow 
    \, \exists 0 < c_1 < c_2 + \infty$ т.ч. 
    \[ c_1 \norm{x}_1 \leq \norm{x_2} \leq C_2 \norm{x}_1\]
    хотя в определении не утверждалось, что одну норму можно оценить через другую
\end{corollary}
 
\begin{proof}
    $X = (X, \norm{\cdot}_1), Y = (X, \norm{\cdot}_2)$ -- как бы 2 разных пространства, но на одном множестве. Рассмотрим оператор
    $Ix = x$. Ясно, что $I \in \Lin(X,Y)$, $I$ -- биекция, $I^{-1} \in \Lin(Y,X)$. Что означает, что $\norm{\cdot}_1$ эквивалентна $\norm{\cdot}_2$?
    $\Leftrightarrow I, I^{-1}$ непрерывны $\Leftrightarrow$ $I$ -- линейный изоморфизм $X$ и $Y$ $\stackrel{\text{т.критерий линейного изоморфизма}}{\Longleftrightarrow} c_1 \norm{x_1} \leq \underbrace{\norm{Ix}_2}_{\norm{x}_2} \leq C_2\norm{x}_1$
\end{proof}
 
Не очень скоро мы получим обобщение этой теоремы. Окажется, что если пространство банахово в обеих нормах, то только одно из последних неравенств влечёт другое.
\begin{statement}
    $(X, \norm{\cdot}), (Y, \norm{\cdot})$ -- линейно изоморфны. Пусть $X$ -- банахово, тогда $Y$ -- банахово.
\end{statement}
\begin{proof}
    \begin{gather*}
        A: X \rightarrow Y \quad A \in \B(X,Y) \quad A \text{ -- линейный изоморфизм }\\
        A^{-1}: Y \rightarrow X \quad A^{-1} \in \B(Y,X) \\
        \seq{y_n}^\infty_{n=1} \text{ -- фундаментальная в } Y \quad x_n = A^{-1} y_n \\
        \norm{x_n - x_m} \leq \norm{A^{-1}} \cdot \norm{y_n - y_m} \Rightarrow \seq{x_n}^\infty_{n=1} \text{ фундаментальная в } X \\
        \intertext{теперь применяем наш, слава богу, непрерывный оператор}
        \Rightarrow \: \exists \liml_{n \to \infty} x_n = x_0 \Rightarrow \liml_{n \to \infty} A x_n = A x_0 \Rightarrow \land \liml_{n \to \infty} y_n = A x_0 \Rightarrow \\
        Y \text{ полное}
    \end{gather*}
\end{proof}
 
\section{Конечномерные пространства}
 
\begin{definition}[Размерность пространства]
    $X$ -- линейное пространство над $\bC$ или $\bR$. Если $\exists n$ линейно независимых 
    элементов в $X$, и $\forall (n+1)$ элементов линейно зависимы, то $\dim X = n$
\end{definition}
 
\begin{definition}
    Если $\forall n \in \bN \: \exists n$ линейно незаисимых элементов, то $X$ -- \textbf{ бесконечномерное}
\end{definition}
 
\begin{theorem}
    $(X, \norm{\cdot}), (Y, \norm{\cdot})$ -- линейные пространства над $\bC, \dim X = \dim Y = n$.
    \[ \Rightarrow X \text{ линейно изоморфно } Y \]
\end{theorem}
 
Поскольку мы обсудили, что изоморфность -- отношение эквивалентности, то можно зафиксировать
\begin{gather*}
    X = l^2_n = \seq{x=(x_1, \ldots, x_n), x_j \in \bC, \norm{X} = \left( \sum^n_{j=1} |x_j|^2\right)^{\frac{1}{2}}}
    \seq{f_j}^n_{j=1} \text{ -- базис в } Y \\
    A : l^2_n \rightarrow Y, A(e_j) = f_j \\
    \intertext{утверждается, что это и будет линейный изоморфизм} 
    A\left(\sum^n_{j=1} x_j e_j\right) = \sum^n_{j=1} x_j f_j, A \in \Lin(l^2_n, Y) \\
    x \in l^2_n, x = \sum^n_{j=1} x_j e_j \\
    \norm{Ax} = \norm{\sum^n_{j=1} x_j f_j} \leq \sum^n_{j=1} |x_j| \norm{f_j} \stackrel{\text{КБШ}}{\leq} \underbrace{\left(\sum^n_{j=1} |x_j|^2 \right)^{\frac{1}{2}}}_{\norm{x}_{l^2_n}} \underbrace{\left( \sum^n_{j=1} \norm{f_j}^2\right)^\frac{1}{2}}_{:= M} \\
    \intertext{мы оценили норму оператора $A$}
    \Rightarrow \norm{Ax}_Y \leq \norm{x}_{l^2_n} \cdot M \Rightarrow A \in \B(l^2_n, Y), \norm{A} \leq M \\
    g(x) := \norm{Ax} \text{ -- функция на } l^2_n \Rightarrow g(x) \text{ -- непрерывна на } l^2_n 
\end{gather*}
 
Теперь рассмортим эту функцию не на всём пространстве, а на единичной сфере $S = \seq{x \in l^2_n, \norm{x}_2 = 1}$ -- компакт в $l^2_n$.
\begin{gather*}
    x \in S, g(x) > 0, g \text{ непрерывная на компакте } S \Rightarrow \\
    \exists x_0 \in S, g(x_0) = \max_{x \in S} \min g(x), r = g(x_0), r > 0 \\
    \text{пусть } x \in l^2_n, x \ne 0 \quad \frac{x}{\norm{x}} \in S \Rightarrow g\left( \frac{x}{\norm{x}} \right) \geq r \Rightarrow \\
    \norm{A\left(\frac{x}{\norm{x}}\right)} \geq r \Rightarrow \norm{Ax} \geq r \norm{x} \: \forall x \in l^2_n \\
    \Rightarrow \text{ -- линейная изометрия}
\end{gather*}
 
\begin{corollary}
    $(X, \norm{\cdot}), \dim X = n \in \bN \Rightarrow$
    \begin{enumerate}
        \item $X$ -- банахово 
        \item $K \subset X, K$ -- относительно компактно $\Leftrightarrow K$ -- ограничено
        \item $K \subset X, K$ -- компакт $\Leftrightarrow K$ -- ограничено и замкнуто  
    \end{enumerate}
\end{corollary}
Мы когда-нибудь выясним, что если в пространстве единичный шар -- компакт, то это пространство конечномерное.
\begin{proof}
    \begin{enumerate}
        \item $l^2_n$ -- полное, $X$ -- линейно изоморфно $l^2_n$ и по утверждению из конца предыдущего параграфа $\Rightarrow l^2_n X$ банахово 
        \item $A \in \B(l^2_n, X), A^{-1} \in \B(X, l^2_n), A, A^{-1}$ -- непрерывны 
        \item аналогично 2
    \end{enumerate}
\end{proof}
 
\begin{corollary}
    $X, \dim X = n, n \in \bN$, на $X$ две нормы $\norm{\cdot}_1, \norm{\cdot}_2 \Rightarrow \norm{\cdot}_1 $ эквивалентна $\norm{\cdot}_2$
\end{corollary}
\begin{proof}
    $(X, \norm{\cdot}_1)$ линейно изоморфно $(X, \norm{\cdot}_2)$.
\end{proof}
 
\begin{theorem}
    $(X, \norm{\cdot}), (Y, \norm{\cdot}), \dim X = n, n \in \bN$
    \[ \Rightarrow \Lin(X,Y) = \B(X,Y) \]
\end{theorem}

\begin{proof}
    Рассмотрим сначала частный случай, потом сведём произвольный случай к частному. Пусть $T \in \Lin(l^2_n, Y)$.
    \begin{gather*}
        e_j = (0, \ldots, 0, \underbrace{1}_j, \ldots, 0) \\
        x \in l^2_n, x = \seq{x_j}^n_{j=1}, x = \sum^n_{j=1} x_j e_j \Rightarrow Tx = \sum^n_{j=1} x_j T e_j
        \intertext{оцениваем норму простейшим образом}
        \norm{Tx} \leq \sum^n_{j=1} |x_j| \cdot \norm{T e_j} \stackrel{\text{КБШ}}{\leq}  \left( \sum^n_{j=1} |x_j|^2 \right)^{\frac{1}{2}} \cdot \underbrace{\left(\sum^n_{j=1} \norm{Te_j}^2 \right)^{\frac{1}{2}}  }_{M} \leq \norm{x}_2 \cdot M \\
        \intertext{2 множитель не зависит от $x$, и раз получилась независимая константа, то оператор непрерывен}
        \Rightarrow T \in \B(l^2_n, Y), \norm{T} \leq M
    \end{gather*}
    теперь произвольный случай, пусть $U \in \Lin(X,Y), \dim X = n$ \\
    \begin{tikzpicture}[baseline= (a).base]
        \node[scale=2] (a) at (0,0){
            \begin{tikzcd}[every arrow/.append style={shift left}, column sep = huge]
                X \arrow{r}{U} \arrow{d}{A^{-1}} & Y\\
                l^2_n \arrow{u}{A} \arrow{ru}{T} &
            \end{tikzcd}
        };
    \end{tikzpicture}

    \begin{gather*}
        A \text{ -- линейный изоморфизм} \\
        T = UA \in \Lin(l^2_n, Y) \stackrel{\text{доказали}}{\Rightarrow} T \in \B(l^2_n, Y) \\
        \Rightarrow U = TA^{-1} \quad A, A^{-1} \text{ непрерывны } \Rightarrow U \in \B(X,Y)
    \end{gather*}
\end{proof}


ранее мы сформулировали следствие, и теперь скажем пару слов о доказательстве
\begin{corollary}
    $(X, \norm{\cdot}_1, \norm{\cdot}_2), \dim X = n < + \infty$ 
    \[ \Rightarrow \norm{\cdot}_1 \text{ эквивалентна } \norm{\cdot}_2 \]
\end{corollary}

\begin{proof}
    $(X = (X, \norm{\cdot}_1)), Y = (X, \norm{\cdot}_2)$ 
    \begin{gather*}
        \begin{cases}
            Ix = x \Rightarrow I \in \Lin(X,Y) \stackrel{\text{теорема}}{\Rightarrow} I \in \B(X,Y) \\
            I^{-1}x=x \quad  I^{-1}: Y \rightarrow X \Rightarrow I^{-1} \in \B(Y,X)
        \end{cases} \Rightarrow \norm{\cdot}_1 \equiv \norm{\cdot}_2 \\
        (\Leftrightarrow \: \exists \: 0 < c_1 < c_2 : c_1 \norm{x}_1 \leq \norm{x_2} \leq c_2 \norm{x_1})
    \end{gather*}
    Если последовательность сходится в одной норме, то под действием непрерывного оператора сходится и в другой.
\end{proof}


Последнее, что хочется сказать в этом параграфеЖ 

\begin{theorem}
    $(X, \norm{\cdot}), \dim X = n < + \infty \Rightarrow$ 
    \[ X^* = \B(X, \bC) \quad \dim X^* = n \]
\end{theorem}

\begin{proof}
    \begin{gather*}
        \B(X, \bC) = \Lin(X, \bC) \\
        \text{пусть } \seq{e_j}^n_{j=1} \text{ -- базис } X, x \in X \Rightarrow x = \sum^n_{j=1} x_j e_j \\
        f_j(x) = x_j, f_j : X \rightarrow \bC, f_j \in \Lin(X, \bC) 
        \intertext{проверим $\seq{f_j}^n_{j=1}$ базис в $X^*$}
        f \in X^*, x = \sum^n_{j=1} x_j e_j \Rightarrow f(x) = \sum^n_{j=1} x_j f(e_j) = \sum^n_{j=1} \alpha_j x_j = \alpha_j f(e_j) \\
        \Rightarrow f(x) \sum^n_{j=1} \alpha_j f_j(x) \: \forall x \in X \\
        \Rightarrow f = \sum^n_{j=1} \alpha_j f_j
    \end{gather*}
    Проверим, что $\seq{f_j}^n_{j=1}$ линейно независимы
    \begin{gather*}
        \text{пусть } \sum^n_{j=1} c_j f_j = \bZero, \text{ то есть } \bZero(x) = 0 \: \forall x \in X \\
        f_j(e_k) = \begin{cases}
            0 & j \ne k \\
            1 & j = k
        \end{cases} \Rightarrow \underbrace{ \left(\sum^n_{j=1} c_j f_j \right) (e_k)}_{=0} = c_k \Rightarrow c_k = 0, 1, \ldots, n \\
        \Rightarrow \seq{f_j}^n_{j=1} \text{ -- базис в } X^*
    \end{gather*}
\end{proof}

Теперь мы расстаёмся с конечномерными пространствами.

\section{Конечномерные подпространства}

Начнём с некоторого общего определения, которое касается метрических пространств.
 
\begin{definition}
    $(X,\rho)$ -- метрическое, $Y \subset X, x_0 \in X, \rho(x_0,Y) = \inf_{y \in Y} \rho(x_0, y)$. Если $\exists y_0 \in Y$ т.ч.$ \rho(x_0, Y) = \rho(x_0,y_0)$, то
    $y_0$ -- \textbf{элемент наилучшего приближения} для $x_0$ в $Y$
\end{definition}

Возникают вопросы, существует ли он, и если да, то единственный ли? Тривиальное замечание
\begin{remark}
    Если $Y$ компакт, то $\exists y_0 \in Y : f(y) = \rho(x_0, y), f(y)$ непрерывна на $Y$. $\exists y_0, f(y_0) = \min_{y \in Y} f(y)$
\end{remark}


теперь мы имеем дело с конечномерным подпространством

\begin{theorem}
    $(X, \norm{\cdot})$ -- нормированное, $L \subset X$. $L$ -- подпространство (в алгебраическом смысле), $\dim L = n < + \infty \Rightarrow$
    \begin{enumerate}
        \item $L$ -- замкнутое \\
        \item $\forall x_0 \in X \: \exists y_0 \in L$ -- элемент наилучшего приближения
    \end{enumerate}
\end{theorem}

\begin{proof}[1]
    Естественно, о компактности никакой речи быть не может, но конечномерность нам поможет. Во-первых, мы уже отмечали, что все конечномерные пространства -- полные. Ещё мы доказывали линейную изоморфность.
Таким образом, $L$ -- полное. А ещё почти на первой лекции мы обсуждали, что если есть полное подмножество метрического пространства, то оно автоматически оказывается замкнутым.
\end{proof}

\begin{proof}[2]
    \begin{gather*}
        \text{пусть } x_0 \in X \setminus L \quad \rho(x_0,L) = d > 0 \\ 
        \rho(x_0,L) = \inf_{y \in L} \norm{x_0 - y} \Rightarrow \: \exists \seq{y_n}^\infty_{n=1}, y_n \in L \\
        \intertext{План такой: мы докажем что последовательность ограниченная, значит, она относительно компактная,
        и из неё можно выбрать сходящуюся подпоследовательность, а так как $L$ замкнуто, то предел будет лежать в $L$. Для оценки воспользуемся неравенством треугольника}
        d < \norm {x_0 - y_n} \leq d + \frac{1}{n}
        \seq{y_n}^\infty_{n=1} \text{ ограничена в } L \\
        \dim L < + \infty \Rightarrow \seq{y_n}^\infty_{n=1} \text{ относительно компактна } \Rightarrow \\
        \exists \seq{n_k}^\infty_{k=1} \: \exists \liml_{k \to \infty} y_{n_k} = y_0, L \text{ -- замкнуто } \Rightarrow y_0 \in L \\
        d \leq \norm{x_0 - y_{n_k}} \leq d + \frac{1}{n_k} \Rightarrow \text{при } k \to \infty \norm{x_0 - y_0} = d
    \end{gather*}
\end{proof}

\begin{remark}
    $\dim L < + \infty$, элемент наименьшего приближения может быть не единственным.
\end{remark}

\begin{example}[$l_2^\infty$]
    $\norm{(x,y)} = \max \seq{|x|, |y|}$. $L = \{(x,y): y = kx, k \ne 0 \}$.
    $(\cdot)$ -- элемент наилучшего приближения, единственный
    \begin{tikzpicture}
        %тут будет рисунок 1
    \end{tikzpicture}

    \begin{tikzpicture}
        % рисунок 2
    \end{tikzpicture}
    все точки будут лежать на одном и том же расстоянии от $(x_1,y_1)$. $\forall x \in [x_1 - y_1, x_1 + y_1], y = 0 \: \forall (\cdot)$ -- элемент наилучшего приближения
\end{example}

\begin{example}[$l^1_2$]
    $\norm{(x, y)}_1 = |x| + |y|$
    %рисунок 3
    $L = \{ (x,y) : y = kx, k \ne \pm 1 \}$, тогда $\exists$ единственный элемент наилучшего приближения
    %рисунок 4
    $L = \{y = x \}$, все точки отрезка -- элементы наилучшего приближения
\end{example}

\begin{example}[$l^2_2$]
    $l_2^2 = \seq{(x,y) : \norm{(x,y)}_2 = \sqrt[root]{|x|^2 + |y|^2}}$
    %рисунок 5
    $\forall L \: \exists$ единственный элемент приближения 
\end{example}

\begin{example}
    $1 < p < + \infty$%
    %рисунок 6,
    $\exists$ единственный элемент. $e_j$ -- шар $l^p_2, 1 < p < + \infty$
\end{example}

\begin{corollary}[про многочлены]
    $C_{\bR}[a,b] = \seq{f: [a,b] \rightarrow \bR}$,
    \begin{gather*}
        \P_n \seq{p(x) = \sum^n_{k=0} a_k x^k, a_k \in \bR} \\
        E_n(f) = \inf_{p \in \P_n} \norm{f - p}_\infty \\
        \Rightarrow \exists p_0 \text{ т.ч. } E_n(f) = \norm{f - p_0}_\infty, p_0 \intertext{носит торжественное название многочлена наилучшего приближения}
    \end{gather*}
\end{corollary}

\begin{proof}
    $\dim \P_n = n+1 \Rightarrow \: \exists p_0$ 
\end{proof}

\begin{remark}
    $\existu p_0$, так как $p_9(x) = 0$ только в $n$ точках 
    В пространстве непрерывных функций единичный шар устроен совершенно кошмарно, хотя норма устроена похожим образом на $l^\infty$. В шаре полно отрезков.
\end{remark}

\section{Конечномерность нормированного пространства с компактным единичным шаром}

\begin{lemma}[Ф.Рисс, о почти перпендикуляре]
    $(X, \norm{\cdot}), L \subsetneq X X, L$ -- замкнутое подпространство, $0 < \varepsilon < 1$
    \[ \Rightarrow \exists x_0, \norm{x_0} = 1, \rho(x_0, L) > 1 - \varepsilon \]
\end{lemma}
Перед доказательсвтом сначала картинка, поясняющая, причём тут <<почти перпендикуляр>>
\begin{tikzpicture}
    %почти перпендикуляр
\end{tikzpicture}
хочется, чтобы был элемент на расстоянии 1. вот хочется чтобы $x_0$ был почти перпендикуляром. 1 обеспечить нельзя, но $1 - \varepsilon$ -- можно.

\begin{proof}
    \begin{gather*}
        z \in X \setminus L, d = \rho(z, L) = \inf_{y \in L} \norm{z - y} \Rightarrow \: \exists y_0 \in L : d \leq \norm{z - y_0} < d(1 + \varepsilon) \\
        x_0 = \frac{z - y_0}{\norm{z-y_0}}, \norm{x_0} = 1
        \intertext{оценим норму разности}
        \text{пусть } y \in L \quad \norm{x_0 -y} = \norm{\frac{z-y_0}{\norm{z-y_0}} - y} = \frac{1}{\norm{z-y_0}} \underbrace{\norm{z - \overbrace{y_0 - y\norm{z-y_0}}_{\in L}}}_{\geq d} \geq \frac{d}{d(1+\varepsilon)} = \frac{1}{1+\varepsilon} \\
        \forall y \in L \Rightarrow \rho(x_0, L) \geq \frac{1}{1 + \varepsilon} > 1 - \varepsilon
    \end{gather*}
\end{proof}

\begin{remark}
    Если $\exists y_0 \in L : \norm{z - y_0} = d$, то $x_0 = \frac{z-y_0}{\norm{z-y_0}} \Rightarrow \rho(x_0, L) = 1$
\end{remark}

\begin{corollary}[из замечания]
    $(X, \norm{\cdot}), L \subsetneq X, L$ -- подпространство, $\dim L < + \infty$ 
    \[ \Rightarrow \exists x_0 \in X \setminus L, \norm{x_0} = 1, \rho(x_0,L) = 1 \] 
\end{corollary}

А это следствие нам понадобится несколько раз.
\begin{corollary}
    $(X, \norm{\cdot}), \seq{L_n}^\infty_{n=1}$, $L_n$ -- замкнутые подпространства. $L_n \subsetneq L_{n+1}, L_1 \ne \varnothing \Rightarrow$
    \[ \exists \seq{y_n}^\infty_{n=1}, y_n = L_n, \rho(y_{n+1}, L_n) \geq \frac{1}{2}, \norm{y_n} = 1 \]
\end{corollary}

\begin{proof}
    пусть $y_1 \in L_1, \norm{y_1} = 1, L_1 \subsetneq L_2 \stackrel{\text{Лемма}}{\Rightarrow} \: \exists y_2 \in L_2, \norm{y_2} = 1$. $\rho(y_2, L_1) \geq \frac{1}{2}$ и так далее по индукции
\end{proof}

\begin{theorem}[Ф.Рисс]
    $(X, \norm{\cdot}), B = \seq{x: \norm{x} < 1}$. $\overline{B} = \seq{x: \norm{x} \leq 1}$
    \[ \overline{B} \text{ -- компакт } \Leftrightarrow \dim X < + \infty \]
\end{theorem}

\begin{proof}
    $\Leftarrow$ уже доказали \\
    $\Rightarrow$ \\
    пусть $\dim X = \infty \Rightarrow \: \exists \seq{x_n}^\infty_{n=1}$ -- линейно независимы
    \begin{gather*}
        L_n = \Lin \seq{x_j}^n_{j=1}, \dim L_n = n, L_n \subsetneq L_{n+1} \\
        \stackrel{\text{Сл.2}}{\Rightarrow} \: \exists \seq{y_n}^\infty_{n=1}, \norm{y_n} = 1, \rho(y_n, L_{n+1}) > \frac{1}{2} \Rightarrow \\
        \norm{y_n-y_m} > \frac{1}{2} \: \forall n, m \Rightarrow \: \nexists \text{ фундаментальной подпоследовательности } \Rightarrow \\
        \nexists \seq{y_{n_k}} : \: \exists \liml_{k \to \infty} y_{n_k} \Rightarrow \overline{B} \text{ не компакт}
    \end{gather*}
\end{proof}

Вот так нам удалось установить, что если в пространстве единичный шар -- компакт, то пространство конечномерное.

\begin{theorem}[о продолжении линейного оператора]
    $(X, \norm{\cdot})$ -- нормированное, $(Y, \norm{\cdot})$ -- банахово, $L \subset X, L$ -- подпространство в алгебраическом смысле
    \[ \overline{L} = X, A \in \B(L,Y) \Rightarrow \: \existu V \in \B(X,Y) : \norm{V}_{\B(X,Y)} = \norm{A}_{\B(L,Y)} \] 
\end{theorem}

\begin{proof}
    Сначала мы должны распростарнить оператор, то есть определить, как он будет действовать на произвольный элемент $X$. Пусть $x \in X$.
    \begin{gather*}
        \exists \seq{x_n}^\infty_{n=1}, x_n \in L, \liml_{n \to \infty} \norm{x-x_n} = 0 \\
        \seq{A x_n}^\infty_{n=1}, A x_n \in Y, \seq{A x_n}^\infty_{n=1} \text{-- фундаментальная  в } Y, \norm{Ax_n - Ax_m} \underset{n, m \to \infty}{\longrightarrow} 0
    \end{gather*}
    Раз последовательность имеет предел, то она фундаментальная. Значит мы не зря в условии требовали банаховость. $Y$ -- банахово, тогда 
    $\exists \liml_{n \to \infty} A x_n \in Y$
    \[ Vx := \liml_{n \to \infty} A X_n \]
    надо убедиться, что определение корректно, то есть что предел не зависит от изначально выбранной последовательности:
    \begin{gather*}
        \text{пусть } \seq{z_n}^\infty_{n=1} \liml_{n \to \infty} z_n = x \Rightarrow \: \exists \liml_{n \to \infty} A z_n \\
        z_n \in L \quad \norm{A x_n - Az_n} \leq \norm{A} \underbrace{\norm{x_n - z_n}_{\underset{n \to \infty}{\longrightarrow} -}} \Rightarrow \liml_{n \to \infty} Ax_n = \liml_{n \to \infty} Az_n 
        \intertext{корректность проверена}
        \text{пусть } x \in L, \text{ пусть } x_n = x \: \forall n \in \bN \Rightarrow Vx = \liml_{n \to \infty} A x_n = Ax \Rightarrow V |_L = A \\
        \text{пусть } \liml_{n \to \infty} x_n = x \Rightarrow x = \liml_{n \to \infty} x_n, Vx = \liml_{n \to \infty} A x_n \Rightarrow \\
        \liml{n \to \infty} \norm{x_n} = \norm{x} \quad \norm{Vx} \leq \liml_{n \to \infty}  \norm{A} \cdot \norm{x_n} = \norm{A} \norm{x} \\
        \Rightarrow \norm{V} \leq \norm{A} \\
        \norm{V} = \sup_{\norm{x} = 1, x \in X} \norm{ Vx} \geq \sup_{{\norm{x} = 1, x \in X}} \norm{Vx} = \norm{A} \\
        \Rightarrow \norm{V} = \norm{A}
    \end{gather*}
\end{proof}


Следующая конструкция, которая ранее упоминалась, это факторпространства.
\section{Факторпространство}

\begin{definition}[класс эквивалентности]
    $X$ -- линейное пространство над $\bC$, $Y$ -- подпространство. $X / Y = \seq{\overline{x}}_{x \in X}$
    \begin{gather*}
        x \sim z \text{ если } x - z \in Y \\
    \overline{x} = \seq{z: z = x + h, h \in Y}  \\
    \overline{x} + \overline{y} = \overline{x+y} \\
    \lambda \in \bC, \lambda \overline{x} = \overline{(\lambda x)} \\
    \varphi: X \Rightarrow X / Y \quad \varphi(x) = \overline{x}
    \end{gather*}
    $\varphi$ -- линейное (канонический гомоморфизм).
\end{definition}

Если пространство будет не замкнутым, то будут ненулевые элементы с нулевой нормой (те, что лежат в замыкании).

\begin{definition}
   $(X, \norm{\cdot})$ -- нормированное, $Y$ -- замкнутое подпространство. $X / Y = \seq{\overline{x}}_{x \in X}$,
    \[ \norm{\overline{x}} = \inf_{z \in \overline{x}} \norm{z} = \inf_{y \in Y} \norm{x-y} = \rho(x,Y)\]
\end{definition}

\begin{theorem}
    $(X, \norm{\cdot}), Y$ -- замкнутое подпространство $\Rightarrow$ 
    \begin{enumerate}
        \item $\norm{\overline{x}}$ в $X / Y$ удовлетворяет аксиомам нормы 
        \item $\varphi: X \rightarrow X/Y, \varphi(x) = \overline{x} \Rightarrow \varphi \in \B(X, X/Y), \norm{\varphi} = 1$
        \item Если $X$ -- банахово, то $X /Y$ -- банахово
    \end{enumerate}
\end{theorem}

\begin{proof}[1]
    \begin{gather*}
        \lambda \in \bC, \lambda \ne 0, x \in X \\
        \norm{\overline{\lambda x}} = \inf_{z \in \overline{x}} \norm{\lambda z} = |\lambda| \inf_{z \in \overline{x}} \norm{z} = |\lambda| \cdot \norm{\overline{x}} \\
        \text{пусть } \overline{x}, \overline{u} \in X / Y, z \in \overline{x}, v \in \overline{y} \\
        \norm{\overline{x} + \overline{u}} \leq \norm{z + v} \leq \norm{z} + \norm{v}  \quad \forall z \in \overline{x}, \forall v \in \overline{u} \\
        \Rightarrow \norm{\overline{x} + \overline{u}} \leq \inf_{z \in \overline{x}} \norm{z} + \inf_{v \in \overline{u}} \norm{v} = \norm{\overline{x}} + \norm{\overline{u}} 
        \intertext{ теперь проверяем в 0, тут как раз нужна замкнутость} 
        \norm{\overline{x}} = 0 \quad \norm{\overline{x}} = \rho(x,Y) = 0 \Rightarrow x \in Y \Rightarrow \overline{x} = Y = \overline{0}
    \end{gather*}
\end{proof}

\begin{proof}[2]
    $\norm{\varphi(x)} = \norm{\overline{x}} = \inf_{z \in \overline{x}} \norm{z} \leq \norm{x} \Rightarrow \varphi \in \B(X, X/Y), \norm{\varphi} \leq 1$. 
    По лемме о почти перпендикуляре, пусть $\varepsilon > 0 \: \exists x_0, \norm{x_0} = 1$
    \begin{gather*}
        \rho(x_0,Y) > 1 - \varepsilon \Rightarrow \norm{\varphi(x_0)} = \rho(x_0, Y) > 1 - \varepsilon \\
        \Rightarrow \norm{\varphi} = \sup_{\seq{x: \norm{x} =1}} \norm{\varphi(x)} > 1 - \varphi \: \forall \varepsilon > 0 \Rightarrow \norm{\varphi} = 1
    \end{gather*}
\end{proof}
\begin{proof}[3]
    Воспользуемся критерием полноты: если сходится ряд из норм, то сходится и сам ряд. $X/Y$ -- полное?
    \begin{gather*}
        \text{пусть } \seq{x_n}^\infty_{n=1} \: \sum^\infty_{n=1} \norm{\overline{x_n}} < + \infty \left( \stackrel{?}{\Rightarrow} \sum^\infty_{n=1} \overline{x_n} \text{сходится в } X/Y \right) \\
        \norm{\overline{x_n}} = \inf_{z \in \overline{x_n}} \norm{z} \Rightarrow \: \exists z_n \in \overline{x_n} : \norm{z_n} \leq 2 \norm{\overline{x_n}} \\
        \Rightarrow \sum^\infty_{n=1} \norm{z_n} < + \infty  X \text{ -- банахово, и по критерию полноты} \Rightarrow \\
        \exists S = \sum^\infty_{n=1} z_n, s \in X 
        \intertext{рассмотрим частичные суммы}
        \begin{cases}
            S_n = \sum^n_{k=1} z_k, \liml_{n \to \infty} s_n = s \\
            \varphi(s_n) = \sum^n_{k=1} \varphi(z_k) = \sum^n_{k=1} \overline{x_k} 
        \end{cases} \varphi \text{ непрерывна} \Rightarrow \\
        \liml_{n \to \infty} \varphi(s_n) = \varphi(s) \in X/Y \\
        \Rightarrow \: \exists \liml_{n \to \infty} \sum^n_{k=1} \overline{x_k} = \sum^\infty_{k=1} \overline{x_k} \Rightarrow X / Y \text{ -- банахово}
    \end{gather*}
\end{proof}

\end{document}