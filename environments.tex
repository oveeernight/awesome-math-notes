\declaretheoremstyle[
spaceabove=6pt, spacebelow=6pt,
headfont=\normalfont\bfseries,
notefont=\mdseries, notebraces={(}{)},
bodyfont=\normalfont,
postheadspace=1em,
qed=\qedsymbol
]{mystyle}


\declaretheorem[name=Теорема, numberwithin=chapter]{theorem}
\declaretheorem[name=Теорема, numbered=no]{theorem*}
\declaretheorem[name=Лемма, numberwithin=chapter]{lemma}
\declaretheorem[name=Лемма, numbered=no]{lemma*}
\declaretheorem[name=Следствие, numberwithin=chapter]{corollary}
\declaretheorem[name=Следствие, numbered=no]{corollary*}
\declaretheorem[name=Пример, numberwithin=chapter]{example}
\declaretheorem[name=Пример, numbered=no]{example*}
\declaretheorem[name=Определение, numberwithin=chapter]{definition}
\declaretheorem[name=Определение, numbered=no]{definition*}
\declaretheorem[name=Замечание, numberwithin=chapter]{remark}
\declaretheorem[name=Замечание, numbered=no]{remark*}
\declaretheorem[name=Доказательство, numbered=no, style=mystyle]{Доказательство}